
\documentclass{article}
%%%%%%%%%%%%%%%%%%%%%%%%%%%%%%%%%%%%%%%%%%%%%%%%%%%%%%%%%%%%%%%%%%%%%%%%%%%%%%%%%%%%%%%%%%%%%%%%%%%%%%%%%%%%%%%%%%%%%%%%%%%%%%%%%%%%%%%%%%%%%%%%%%%%%%%%%%%%%%%%%%%%%%%%%%%%%%%%%%%%%%%%%%%%%%%%%%%%%%%%%%%%%%%%%%%%%%%%%%%%%%%%%%%%%%%%%%%%%%%%%%%%%%%%%%%%
%TCIDATA{OutputFilter=LATEX.DLL}
%TCIDATA{Version=5.00.0.2552}
%TCIDATA{<META NAME="SaveForMode" CONTENT="1">}
%TCIDATA{Created=Saturday, September 19, 2015 08:17:34}
%TCIDATA{LastRevised=Monday, September 21, 2015 22:32:42}
%TCIDATA{<META NAME="GraphicsSave" CONTENT="32">}
%TCIDATA{<META NAME="DocumentShell" CONTENT="Scientific Notebook\Blank Document">}
%TCIDATA{CSTFile=Math with theorems suppressed.cst}
%TCIDATA{PageSetup=72,72,72,72,0}
%TCIDATA{ComputeDefs=
%$f(a-\epsilon )=f(a$
%}

%TCIDATA{AllPages=
%F=36,\PARA{038<p type="texpara" tag="Body Text" >\hfill \thepage}
%}


\newtheorem{theorem}{Theorem}
\newtheorem{acknowledgement}[theorem]{Acknowledgement}
\newtheorem{algorithm}[theorem]{Algorithm}
\newtheorem{axiom}[theorem]{Axiom}
\newtheorem{case}[theorem]{Case}
\newtheorem{claim}[theorem]{Claim}
\newtheorem{conclusion}[theorem]{Conclusion}
\newtheorem{condition}[theorem]{Condition}
\newtheorem{conjecture}[theorem]{Conjecture}
\newtheorem{corollary}[theorem]{Corollary}
\newtheorem{criterion}[theorem]{Criterion}
\newtheorem{definition}[theorem]{Definition}
\newtheorem{example}[theorem]{Example}
\newtheorem{exercise}[theorem]{Exercise}
\newtheorem{lemma}[theorem]{Lemma}
\newtheorem{notation}[theorem]{Notation}
\newtheorem{problem}[theorem]{Problem}
\newtheorem{proposition}[theorem]{Proposition}
\newtheorem{remark}[theorem]{Remark}
\newtheorem{solution}[theorem]{Solution}
\newtheorem{summary}[theorem]{Summary}
\newenvironment{proof}[1][Proof]{\noindent\textbf{#1.} }{\ \rule{0.5em}{0.5em}}


\begin{document}


$\bigskip $Problem2.14

$\FRAME{itbpF}{4.5in}{0.9in}{0in}{}{}{Figure}{\special{language "Scientific
Word";type "GRAPHIC";maintain-aspect-ratio TRUE;display "USEDEF";valid_file
"T";width 4.5in;height 0.9in;depth 0in;original-width
11.2106in;original-height 2.2373in;cropleft "0";croptop "1";cropright
"1";cropbottom "0";tempfilename 'NUWJ9101.wmf';tempfile-properties "XPR";}}$

$\int_{0}^{\infty }p(t)\chi _{\lbrack u,+\infty )}(t)dt=1-F\left( u\right) ,$

$\int_{0}^{\infty }[1-F(u)]du=\int_{0}^{\infty }\left[ \int_{0}^{\infty
}p(t)\chi _{\lbrack u,+\infty )}(t)dt\right] du,$

changing the order of integration,

=$\int_{0}^{\infty }p(t)dt\int_{0}^{\infty }\chi _{\lbrack u,+\infty
)}(t)du=\int_{0}^{\infty }p(t)dt\int_{0}^{t}du=\int_{0}^{\infty }tp(t)dt=EX$

Probelm2.15 Let X and Y be two random variables and define $X\wedge Y=\min
\left( X,Y\right) ,X\vee Y=\max \left( X,Y\right) .$

Show that $E\left( X\vee Y\right) =E\left( X\right) +E\left( Y\right)
-E\left( X\wedge Y\right) .$

Proof: $X\vee Y=\frac{\left\vert X-Y\right\vert +\left( X+Y\right) }{2}%
,X\wedge Y=\frac{\left( X+Y\right) -\left\vert X-Y\right\vert }{2},$%
therefore $X\vee Y=X+Y-\left( X+Y\right) ,$

By the linearity of $E\left( \cdot \right) ,E\left( X\vee Y\right) =E\left(
X\right) +E\left( Y\right) -E\left( X\wedge Y\right) .$

2.17(a) $\frac{1}{2^{\frac{1}{3}}}(b)0$

2.18Let $F(a)=\int_{R}\left\vert x-a\right\vert p(x)dx=\int_{a}^{\infty
}(x-a)p(x)dx+\int_{-\infty }^{a}(a-x)p(x)dx,$

$F^{\prime }(a)=\int_{a}^{\infty }(-p(x))dx+\int_{-\infty }^{a}p(x)dx,$

Let $F^{\prime }(a)=0,$combined with $\int_{a}^{\infty }p(x)dx+\int_{-\infty
}^{a}p(x)dx=1,$

we have $\int_{a}^{\infty }p(x)dx=\int_{-\infty }^{a}p(x)dx=\frac{1}{2},$

that is $a=m\left( \text{median}\right) $

Problem 2.19 Prove that $\frac{d}{da}E\left( X-a\right) ^{2}=0\iff E\left(
X\right) =a.$

Proof:$E\left( X-a\right) ^{2}=a^{2}-2E\left( X\right) a+E\left(
X^{2}\right) ;$

$\frac{d}{da}E\left( X-a\right) ^{2}=0\iff 2a-2E\left( X\right) =0\iff
E\left( X\right) =a.$

As a parabola with positive orientation, $a=E\left( X\right) $ is indeed a
minimum.

Assumptions:$x^{2}f_{X}\left( x\right) $ should be square integrable to make 
$E\left( X^{2}\right) $ meaningful. 

2.22 $EX=\frac{2\beta }{\sqrt{\pi }};DX=\frac{-8+3\pi }{2\pi }\beta ^{2}$

2.26(1) symmetric pdf: normal distribution, beta distribution for parameter $%
\alpha =\beta $ and uniform distribution.

\qquad \qquad (5) median=0.69,mean=1, for exponential distribution

2.27(1) normal distribution\qquad \qquad

\qquad \qquad (2) uniform distribution

\qquad \qquad (3) Suppose the mode lies at $a-\epsilon ,$since pdf decreases
on the right side of this point, we have

$f(a-\epsilon )\geq f(a)\geq f(a+\epsilon ),$ but $f(a-\epsilon
)=f(a+\epsilon ),$ hence $f(x)=const$ in the invertal $\left[ a-\epsilon
,a+\epsilon \right] .$

By the definition it is easy to verify that the point of symmetry is a mode.

\qquad \qquad (4) mode 0

2.28 $\left( a\right) $Show that if a pdf is symmetric about a point $a$,
then $\alpha _{3}=\frac{\mu _{3}}{\left( \mu _{2}\right) ^{3/2}}=0.$

Proof: Let $p\left( x\right) $ be the pdf of a random variable $X,$which
satisfies $p\left( x\right) =p\left( 2a-x\right) $ for any $x\in R.$

$\mu _{3}=E\left[ X-E\left( X\right) \right] ^{3}.E\left( X\right)
=\int_{-\infty }^{\infty }xp\left( x\right) dx=\int_{-\infty }^{\infty
}\left( x-a\right) p\left( x\right) dx+\int_{-\infty }^{\infty }ap\left(
x\right) dx$

$=\int_{a}^{\infty }(x-a)p(x)dx+\int_{-\infty }^{a}(x-a)p(x)dx+a,$ let $%
t=2a-x$ in the second integral,

=$\int_{a}^{\infty }(x-a)p(x)dx+\int_{a}^{\infty }(a-t)p(t)dt+a$

$=a,$

Let $Y=X-E\left( X\right) =X-a,$then $p_{Y}\left( y\right) =p_{X}\left(
y+a\right) .$Therefore $p_{Y}\left( y\right) $ satisfies $p\left( y\right)
=p_{X}\left( y+a\right) =p_{X}\left( 2a-\left( y+a\right) \right)
=p_{X}\left( a-y\right) =$ $p_{Y}\left( -y\right) .$ That is, $p_{Y}\left(
y\right) $ is even.

$E\left[ X-E\left( X\right) \right] ^{3}=E\left( Y^{3}\right) =0$ since
integrating every even function on a symmetric interval about the origin
gives 0.

\qquad \qquad\ (b) 2

\qquad \qquad\ (c) for $f(x)=\frac{1}{\sqrt{2\pi }}e^{-\frac{x^{2}}{2}%
},\alpha _{4}=3;$

\qquad \qquad \qquad \qquad for $f(x)=\frac{1}{2}\chi _{(-1,1)},\alpha _{4}=%
\frac{3}{5};$

\qquad \qquad \qquad \qquad for $f(x)=\frac{1}{2}e^{-\left\vert x\right\vert
},\alpha _{4}=6.$

peak degree:$\frac{1}{2}e^{-\left\vert x\right\vert }\geq \frac{1}{\sqrt{%
2\pi }}e^{-\frac{x^{2}}{2}}\geq \frac{1}{2}\chi _{(-1,1)}.$

Problem 2.36 Prove that a pdf of lognormal distribution does not have a
moment generating function,

that is 

$M_{X}\left( t\right) =\int_{0}^{\infty }\frac{e^{tx}}{\sqrt{2\pi }x}%
e^{-\left( \log x\right) ^{2}/2}dx.$

For $t\neq 0,\frac{e^{tx}}{\sqrt{2\pi }x}e^{-\left( \log x\right) ^{2}/2}=%
\frac{1}{\sqrt{2\pi }}e^{tx-\frac{\left( \log x\right) ^{2}}{2}-\log x}>M,$
for sufficient large $x.$

Hence the improper integral does not converge, which means $M_{X}\left(
t\right) $ does not exist.

\end{document}
