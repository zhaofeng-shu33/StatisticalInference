
\documentclass{article}
\usepackage{amsmath}

%%%%%%%%%%%%%%%%%%%%%%%%%%%%%%%%%%%%%%%%%%%%%%%%%%%%%%%%%%%%%%%%%%%%%%%%%%%%%%%%%%%%%%%%%%%%%%%%%%%%%%%%%%%%%%%%%%%%%%%%%%%%%%%%%%%%%%%%%%%%%%%%%%%%%%%%%%%%%%%%%%%%%%%%%%%%%%%%%%%%%%%%%%%%%%%%%%%%%%%%%%%%%%%%%%%%%%%%%%%%%%%%%%%%
%TCIDATA{OutputFilter=LATEX.DLL}
%TCIDATA{Version=5.00.0.2552}
%TCIDATA{<META NAME="SaveForMode" CONTENT="1">}
%TCIDATA{Created=Friday, October 02, 2015 13:50:43}
%TCIDATA{LastRevised=Saturday, October 03, 2015 00:15:05}
%TCIDATA{<META NAME="GraphicsSave" CONTENT="32">}
%TCIDATA{<META NAME="DocumentShell" CONTENT="Scientific Notebook\Blank Document">}
%TCIDATA{CSTFile=Math with theorems suppressed.cst}
%TCIDATA{PageSetup=72,72,72,72,0}
%TCIDATA{AllPages=
%F=36,\PARA{038<p type="texpara" tag="Body Text" >\hfill \thepage}
%}


\newtheorem{theorem}{Theorem}
\newtheorem{acknowledgement}[theorem]{Acknowledgement}
\newtheorem{algorithm}[theorem]{Algorithm}
\newtheorem{axiom}[theorem]{Axiom}
\newtheorem{case}[theorem]{Case}
\newtheorem{claim}[theorem]{Claim}
\newtheorem{conclusion}[theorem]{Conclusion}
\newtheorem{condition}[theorem]{Condition}
\newtheorem{conjecture}[theorem]{Conjecture}
\newtheorem{corollary}[theorem]{Corollary}
\newtheorem{criterion}[theorem]{Criterion}
\newtheorem{definition}[theorem]{Definition}
\newtheorem{example}[theorem]{Example}
\newtheorem{exercise}[theorem]{Exercise}
\newtheorem{lemma}[theorem]{Lemma}
\newtheorem{notation}[theorem]{Notation}
\newtheorem{problem}[theorem]{Problem}
\newtheorem{proposition}[theorem]{Proposition}
\newtheorem{remark}[theorem]{Remark}
\newtheorem{solution}[theorem]{Solution}
\newtheorem{summary}[theorem]{Summary}
\newenvironment{proof}[1][Proof]{\noindent\textbf{#1.} }{\ \rule{0.5em}{0.5em}}


\begin{document}


\FRAME{ftbpF}{4.5in}{0.9in}{0in}{}{}{Figure}{\special{language "Scientific
Word";type "GRAPHIC";maintain-aspect-ratio TRUE;display "USEDEF";valid_file
"T";width 4.5in;height 0.9in;depth 0in;original-width
8.0436in;original-height 1.9311in;cropleft "0";croptop "1";cropright
"1";cropbottom "0";tempfilename 'NVLIT502.wmf';tempfile-properties "XPR";}}

Proof of Conclusion in P220 Textbook

$\left( n-1\right) S_{n}^{2}=\overset{n}{\underset{i=1}{\sum }}\left( X_{i}-%
\overline{X}_{n}\right) ^{2}=\overset{n}{\underset{i=1}{\sum }}X_{i}^{2}-n%
\overline{X}_{n}^{2}$

$\left( n-2\right) S_{n-1}^{2}=\overset{n-1}{\underset{i=1}{\sum }}\left(
X_{i}-\overline{X}_{n-1}\right) ^{2}=\overset{n-1}{\underset{i=1}{\sum }}%
X_{i}^{2}-\left( n-1\right) \overline{X}_{n-1}^{2}$

$\left( n-1\right) S_{n}^{2}-\left( n-2\right) S_{n-1}^{2}=X_{n}^{2}-n%
\overline{X}_{n}^{2}+\left( n-1\right) \overline{X}_{n-1}^{2}$

=$X_{n}^{2}-\frac{1}{n}\left( \overset{n}{\underset{i=1}{\sum }}X_{i}\right)
^{2}+\left( n-1\right) \overline{X}_{n-1}^{2}$

=$X_{n}^{2}-\frac{1}{n}\left( \left( n-1\right) \overline{X}%
_{n-1}+X_{n}\right) ^{2}+\left( n-1\right) \overline{X}_{n-1}^{2}$

=$\left( 1-\frac{1}{n}\right) X_{n}^{2}-\frac{2}{n}\left( n-1\right) 
\overline{X}_{n-1}X_{n}+\left( n-1\right) \left( 1-\frac{n-1}{n}\right) 
\overline{X}_{n-1}^{2}$

=$\frac{n-1}{n}\left( X_{n}^{2}-2\overline{X}_{n-1}X_{n}+\overline{X}%
_{n-1}^{2}\right) =\frac{n-1}{n}\left( X_{n}-\overline{X}_{n-1}\right) ^{2}$

\FRAME{ftbpF}{4.5in}{0.9in}{0in}{}{}{Figure}{\special{language "Scientific
Word";type "GRAPHIC";maintain-aspect-ratio TRUE;display "USEDEF";valid_file
"T";width 4.5in;height 0.9in;depth 0in;original-width
10.1001in;original-height 1.9873in;cropleft "0";croptop "1";cropright
"1";cropbottom "0";tempfilename 'NVLIT503.wmf';tempfile-properties "XPR";}}

Calculate the pdf of t-distribution: $\left( \text{let }p=n-1\right) $

the joint pdf of $U,V$ goes like $p\left( x,y\right) =\frac{1}{\sqrt{2\pi }}%
e^{-\frac{x^{2}}{2}}\frac{1}{2^{p}\Gamma \left( \frac{p}{2}\right) }y^{\frac{%
p}{2}-1}e^{-\frac{y}{2}}.$

Making transformation $u=x/\sqrt{y/p},v=y$ gives the joint pdf of $u,v$

$p\left( u,v\right) =\frac{1}{\sqrt{2\pi }}e^{-\frac{u^{2}v}{2p}}\frac{1}{%
2^{p}\Gamma \left( \frac{p}{2}\right) }v^{\frac{p}{2}-1}e^{-\frac{v}{2}}%
\sqrt{\frac{v}{p}}.$

The maginal pdf $p\left( u\right) =\int_{0}^{\infty }\frac{1}{\sqrt{2\pi }}%
e^{-\frac{u^{2}v}{2p}}\frac{1}{2^{p/2}\Gamma \left( \frac{p}{2}\right) }v^{%
\frac{p}{2}-1}e^{-\frac{v}{2}}\sqrt{\frac{v}{p}}dv$

=$\frac{1}{2^{p/2}\Gamma \left( \frac{p}{2}\right) \sqrt{2\pi p}}%
\int_{0}^{\infty }v^{\frac{p-1}{2}}e^{-\frac{v}{2}\left( 1+\frac{u^{2}}{p}%
\right) }dv=\frac{1}{2^{p/2}\Gamma \left( \frac{p}{2}\right) \sqrt{2\pi p}}%
\frac{\int_{0}^{\infty }v^{\frac{p-1}{2}}e^{-y}dv}{\left( \frac{1}{2}\left(
1+\frac{u^{2}}{p}\right) \right) ^{\frac{p+1}{2}}}=\frac{1}{2^{p/2}\Gamma
\left( \frac{p}{2}\right) \sqrt{2\pi p}}\frac{2^{\frac{p+1}{2}}\Gamma \left( 
\frac{p+1}{2}\right) }{\left( 1+\frac{u^{2}}{p}\right) ^{\frac{p+1}{2}}}$

=$\frac{\Gamma \left( \frac{p+1}{2}\right) }{\Gamma \left( \frac{p}{2}%
\right) \sqrt{\pi p}}\left( 1+\frac{u^{2}}{p}\right) ^{-\frac{p+1}{2}}.$

If $p=1;$

$p\left( u\right) =\frac{1}{\pi \left( 1+u^{2}\right) }\symbol{126}%
Cauchy(1,0).$

$T_{p}\symbol{126}t_{p},$if $p>1$

$E\left( T_{p}\right) =\int_{-\infty }^{\infty }\frac{\Gamma \left( \frac{p+1%
}{2}\right) u}{\Gamma \left( \frac{p}{2}\right) \sqrt{\pi p}}\left( 1+\frac{%
u^{2}}{p}\right) ^{-\frac{p+1}{2}}du=0.$

If $p>2,$

$D\left( T_{p}\right) =E\left( T_{p}^{2}\right) =\int_{-\infty }^{\infty }%
\frac{\Gamma \left( \frac{p+1}{2}\right) u^{2}}{\Gamma \left( \frac{p}{2}%
\right) \sqrt{\pi p}}\left( 1+\frac{u^{2}}{p}\right) ^{-\frac{p+1}{2}}du$

=$\frac{\Gamma \left( \frac{p+1}{2}\right) }{\Gamma \left( \frac{p}{2}%
\right) \sqrt{\pi p}}\int_{-\infty }^{\infty }\frac{pu}{2}\frac{1}{1-\frac{%
p+1}{2}}\frac{d}{du}\left( 1+\frac{u^{2}}{p}\right) ^{-\frac{p+1}{2}+1}du$

=$\frac{\Gamma \left( \frac{p+1}{2}\right) }{\Gamma \left( \frac{p}{2}%
\right) \sqrt{\pi p}}\int_{-\infty }^{\infty }\frac{p}{2}\frac{2}{p-1}\left(
1+\frac{u^{2}}{p}\right) ^{-\frac{p-1}{2}}du$

=$\frac{2p}{p-1}\frac{\Gamma \left( \frac{p+1}{2}\right) }{\Gamma \left( 
\frac{p}{2}\right) \sqrt{\pi p}}\int_{0}^{\infty }\left( 1+\frac{u^{2}}{p}%
\right) ^{-\frac{p-1}{2}}du$

=$\frac{2p}{p-1}\frac{\Gamma \left( \frac{p+1}{2}\right) }{\Gamma \left( 
\frac{p}{2}\right) \sqrt{\pi }}\int_{0}^{\infty }\left( 1+u^{2}\right) ^{-%
\frac{p-1}{2}}du$

=$\frac{2p}{p-1}\frac{\Gamma \left( \frac{p+1}{2}\right) }{\Gamma \left( 
\frac{p}{2}\right) \sqrt{\pi }}\int_{0}^{\frac{\pi }{2}}\left( \cos u\right)
^{p-3}du$

=$\frac{2p}{p-1}\frac{\Gamma \left( \frac{p+1}{2}\right) }{\Gamma \left( 
\frac{p}{2}\right) \sqrt{\pi }}\int_{0}^{\frac{\pi }{2}}\left( \cos u\right)
^{p-3}du$

Recall that $B\left( a,b\right) =\int_{0}^{1}x^{a-1}\left( 1-x\right)
^{b-1}dx,$ Let $x=\cos ^{2}\left( u\right) ,$

$B\left( a,b\right) =2\int_{0}^{\frac{\pi }{2}}\cos ^{2a-1}\left( u\right)
\sin ^{2b-1}\left( u\right) dx,$Let $2a-1=p-3,2b-1=0$

$\left( p-3\geq 0,\text{making }a\text{ within domain}\right) $

$B\left( \frac{p}{2}-1,\frac{1}{2}\right) =2\int_{0}^{\frac{\pi }{2}}\cos
^{p-3}\left( u\right) dx$

Hence the above equality $\frac{2p}{p-1}\frac{\Gamma \left( \frac{p+1}{2}%
\right) }{\Gamma \left( \frac{p}{2}\right) \sqrt{\pi }}\int_{0}^{\frac{\pi }{%
2}}\left( \cos u\right) ^{p-3}du$

$=\frac{2p}{p-1}\frac{\Gamma \left( \frac{p+1}{2}\right) }{\Gamma \left( 
\frac{p}{2}\right) \sqrt{\pi }}\frac{B\left( \frac{p}{2}-1,\frac{1}{2}%
\right) }{2}=\frac{p}{p-1}\frac{\Gamma \left( \frac{p+1}{2}\right) }{\Gamma
\left( \frac{p}{2}\right) \sqrt{\pi }}\frac{\Gamma \left( \frac{p}{2}%
-1\right) \sqrt{\pi }}{\Gamma \left( \frac{p-1}{2}\right) }=\frac{p}{p-1}%
\frac{\frac{p-1}{2}}{\frac{p}{2}-1}=\frac{p}{p-2}.$

The relation between sample Mean and Variance for non-normal samples:

$D\left( S^{2}\right) =$

\end{document}
