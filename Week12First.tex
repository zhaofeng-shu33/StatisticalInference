
\documentclass{article}
\usepackage{amsmath}

%%%%%%%%%%%%%%%%%%%%%%%%%%%%%%%%%%%%%%%%%%%%%%%%%%%%%%%%%%%%%%%%%%%%%%%%%%%%%%%%%%%%%%%%%%%%%%%%%%%%%%%%%%%%%%%%%%%%%%%%%%%%%%%%%%%%%%%%%%%%%%%%%%%%%%%%%%%%%%%%%%%%%%%%%%%%%%%%%%%%%%%%%%%%%%%%%%%%%%%%%%%%%%%%%%%%%%%%%%%%%%%%%%%%
%TCIDATA{OutputFilter=LATEX.DLL}
%TCIDATA{Version=5.00.0.2552}
%TCIDATA{<META NAME="SaveForMode" CONTENT="1">}
%TCIDATA{Created=Friday, December 04, 2015 00:33:47}
%TCIDATA{LastRevised=Friday, December 04, 2015 13:05:11}
%TCIDATA{<META NAME="GraphicsSave" CONTENT="32">}
%TCIDATA{<META NAME="DocumentShell" CONTENT="Scientific Notebook\Blank Document">}
%TCIDATA{CSTFile=Math with theorems suppressed.cst}
%TCIDATA{PageSetup=72,72,72,72,0}
%TCIDATA{AllPages=
%F=36,\PARA{038<p type="texpara" tag="Body Text" >\hfill \thepage}
%}


\newtheorem{theorem}{Theorem}
\newtheorem{acknowledgement}[theorem]{Acknowledgement}
\newtheorem{algorithm}[theorem]{Algorithm}
\newtheorem{axiom}[theorem]{Axiom}
\newtheorem{case}[theorem]{Case}
\newtheorem{claim}[theorem]{Claim}
\newtheorem{conclusion}[theorem]{Conclusion}
\newtheorem{condition}[theorem]{Condition}
\newtheorem{conjecture}[theorem]{Conjecture}
\newtheorem{corollary}[theorem]{Corollary}
\newtheorem{criterion}[theorem]{Criterion}
\newtheorem{definition}[theorem]{Definition}
\newtheorem{example}[theorem]{Example}
\newtheorem{exercise}[theorem]{Exercise}
\newtheorem{lemma}[theorem]{Lemma}
\newtheorem{notation}[theorem]{Notation}
\newtheorem{problem}[theorem]{Problem}
\newtheorem{proposition}[theorem]{Proposition}
\newtheorem{remark}[theorem]{Remark}
\newtheorem{solution}[theorem]{Solution}
\newtheorem{summary}[theorem]{Summary}
\newenvironment{proof}[1][Proof]{\noindent\textbf{#1.} }{\ \rule{0.5em}{0.5em}}
\input{tcilatex}

\begin{document}


8.11

the posterior distribution of $\sigma \symbol{126}IG\left( \alpha +\frac{n-1%
}{2},\left( \frac{n-1}{2}s^{2}+\frac{1}{\beta }\right) ^{-1}\right) $

\bigskip $P\left( \sigma \leq 1|\qquad s^{2}\right) =P\left( \sigma ^{2}\leq
1|\qquad s^{2}\right) =\frac{\int_{0}^{1}\frac{1}{\left( \sigma ^{2}\right)
^{\alpha +\frac{n+1}{2}}}e^{-\frac{\left( \frac{1}{\beta }+\frac{\left(
n-1\right) }{2}s^{2}\right) }{\sigma ^{2}}}d\sigma ^{2}}{\Gamma \left(
\alpha +\frac{n-1}{2}\right) \left( \frac{1}{\beta }+\frac{\left( n-1\right) 
}{2}s^{2}\right) ^{\alpha +\frac{n-1}{2}}}$

$=\frac{\int_{\frac{1}{\beta }+\frac{\left( n-1\right) }{2}s^{2}}^{\infty
}x^{\alpha +\frac{n-3}{2}}e^{-x}dx}{\Gamma \left( \alpha +\frac{n-1}{2}%
\right) \left( \frac{1}{\beta }+\frac{\left( n-1\right) }{2}s^{2}\right)
^{2\alpha +n-1}},$the exact solution is hard to get.

The required region is $\left\{ \vec{x}|P\left( \sigma \leq 1|\qquad
s^{2}\left( \vec{x}\right) \right) >\frac{1}{2}\right\} .$

$\left( b\right) LRT:L\left( \sigma ^{2}|s^{2}\right) =\frac{n-1}{\sigma ^{2}%
}\frac{1}{2^{\frac{n-1}{2}}\Gamma \left( \frac{n-1}{2}\right) }\left( \frac{%
\left( n-1\right) s^{2}}{\sigma ^{2}}\right) ^{\frac{n-1}{2}-1}e^{-\frac{n-1%
}{2\sigma ^{2}}s^{2}}$

$\underset{\sigma \in R}{\sup }L\left( \sigma ^{2}|s^{2}\right) =L\left(
\sigma ^{2}|s^{2}\right) _{\sigma ^{2}=s^{2}}=\frac{1}{s^{2}}\frac{1}{2^{%
\frac{n-1}{2}}\Gamma \left( \frac{n-1}{2}\right) }\left( n-1\right) ^{\frac{%
n-1}{2}}e^{-\frac{n-1}{2}}$

$L\left( \sigma ^{2}|s^{2}\right) =\QATOPD\{ . {1,s^{2}\leq 1}{\frac{%
Exp\left( -\left( n-1\right) \left( s^{2}-1\right) /2\right) }{\left(
s^{2}\right) ^{\left( n-1\right) /2}},s^{2}>1}.$ It is hard to find $\alpha $
and $\beta ,s.t.$the two inequalities $\frac{Exp\left( -\left( n-1\right)
\left( s^{2}-1\right) /2\right) }{\left( s^{2}\right) ^{\left( n-1\right) /2}%
}<\frac{1}{2}$and $\frac{\int_{\frac{1}{\beta }+\frac{\left( n-1\right) }{2}%
s^{2}}^{\infty }x^{\alpha +\frac{n-3}{2}}e^{-x}dx}{\Gamma \left( \alpha +%
\frac{n-1}{2}\right) \left( \frac{1}{\beta }+\frac{\left( n-1\right) }{2}%
s^{2}\right) ^{2\alpha +n-1}}>\frac{1}{2}$ are equivalent. 

\bigskip 

8.12

For all example we choose $\sigma =1$

$\left( a\right) $ From example 8.3.3 the power function $\beta \left( \mu
\right) =1-\Phi \left( \left( c-\mu \right) \sqrt{n}\right) ,$ $\Phi $ is
the cdf of standard norm distribution and $c$ is determined by $\underset{%
\mu \leq 0}{\sup }\beta \left( \mu \right) =\alpha =0.05,$for different $n,$
we can calculate $c$ and plot the power function, see the graphs in word
document.

$\left( b\right) $ Similarly we can calculate the power function for this
case is 

$\beta \left( \mu \right) =P_{\mu }\left( X\in R\right) ,$where $R=\left\{ 
\vec{x}|\left\vert \bar{x}\right\vert >c\right\} $

$P_{\mu }\left( X\in R\right) =P_{\mu }\left( \left\vert \overline{X}%
\right\vert >c\right) $

$=P_{\mu }\left( \sqrt{n}\left( \overline{X}-\mu \right) >\sqrt{n}\left(
c-\mu \right) \right) +P_{\mu }\left( \sqrt{n}\left( \overline{X}-\mu
\right) <\sqrt{n}\left( -c-\mu \right) \right) $

$\sqrt{n}\left( \overline{X}-\mu \right) \symbol{126}N\left( 0,1\right)
\implies \beta \left( \mu \right) =1-\Phi \left( \sqrt{n}\left( c-\mu
\right) \right) +\Phi \left( \sqrt{n}\left( -c-\mu \right) \right) $

$\beta \left( 0\right) =\alpha =0.05,$

$\implies 2\Phi \left( -\sqrt{n}c\right) =0.05,$for different $n,$ we can
calculate $c$ and plot the power function, see the graphs in word document.

\end{document}
