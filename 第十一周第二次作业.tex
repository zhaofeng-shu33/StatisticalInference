
\documentclass{ctexart}
%%%%%%%%%%%%%%%%%%%%%%%%%%%%%%%%%%%%%%%%%%%%%%%%%%%%%%%%%%%%%%%%%%%%%%%%%%%%%%%%%%%%%%%%%%%%%%%%%%%%%%%%%%%%%%%%%%%%%%%%%%%%%%%%%%%%%%%%%%%%%%%%%%%%%%%%%%%%%%%%%%%%%%%%%%%%%%%%%%%%%%%%%%%%%%%%%%%%%%%%%%%%%%%%%%%%%%%%%%%%%%%%%%%%%%%%%%%%%%%%%%%%%%%%%%%%
\usepackage{amssymb}
\usepackage{amsmath}
\def\QATOPD#1#2#3#4{{#3 \atopwithdelims#1#2 #4}}%

\setcounter{MaxMatrixCols}{10}
%TCIDATA{OutputFilter=LATEX.DLL}
%TCIDATA{Version=5.00.0.2552}
%TCIDATA{<META NAME="SaveForMode" CONTENT="1">}
%TCIDATA{Created=Sunday, November 29, 2015 17:26:26}
%TCIDATA{LastRevised=Monday, November 30, 2015 17:07:34}
%TCIDATA{<META NAME="GraphicsSave" CONTENT="32">}
%TCIDATA{<META NAME="DocumentShell" CONTENT="Standard LaTeX\Blank - Standard LaTeX Article">}
%TCIDATA{CSTFile=40 LaTeX article.cst}

\newtheorem{theorem}{Theorem}
\newtheorem{acknowledgement}[theorem]{Acknowledgement}
\newtheorem{algorithm}[theorem]{Algorithm}
\newtheorem{axiom}[theorem]{Axiom}
\newtheorem{case}[theorem]{Case}
\newtheorem{claim}[theorem]{Claim}
\newtheorem{conclusion}[theorem]{Conclusion}
\newtheorem{condition}[theorem]{Condition}
\newtheorem{conjecture}[theorem]{Conjecture}
\newtheorem{corollary}[theorem]{Corollary}
\newtheorem{criterion}[theorem]{Criterion}
\newtheorem{definition}[theorem]{Definition}
\newtheorem{example}[theorem]{Example}
\newtheorem{exercise}[theorem]{Exercise}
\newtheorem{lemma}[theorem]{Lemma}
\newtheorem{notation}[theorem]{Notation}
\newtheorem{problem}[theorem]{Problem}
\newtheorem{proposition}[theorem]{Proposition}
\newtheorem{remark}[theorem]{Remark}
\newtheorem{solution}[theorem]{Solution}
\newtheorem{summary}[theorem]{Summary}
\newenvironment{proof}[1][Proof]{\noindent\textbf{#1.} }{\ \rule{0.5em}{0.5em}}


\begin{document}


8.1 $X_{1},..X_{1000}\symbol{126}Bernoulli\left( p\right) ,0<p<1,$零%
假设$H_{0}:p=\frac{1}{2},$对$H_{1}:p\neq \frac{1}{2}$

$\Sigma X_{i}\symbol{126}\left( 1000,p\right) $:由LME 知$p=%
\overline{X}$可以极大化该样本%
的likelihood function,即LRT statistic

$\lambda \left( \Sigma x_{i}\right) =$ $\frac{\binom{1000}{\Sigma x_{i}}%
\left( \frac{1}{2}\right) ^{1000}}{\binom{1000}{\Sigma x_{i}}\bar{x}^{\Sigma
x_{i}}\left( 1-\bar{x}\right) ^{1000-\Sigma x_{i}}}=$ $\frac{\left( \frac{1}{%
2}\right) ^{1000}}{\bar{x}^{\Sigma x_{i}}\left( 1-\bar{x}\right)
^{1000-\Sigma x_{i}}},$设置Rejection region by LRT method of
sufficient statistic:$R=\left\{ \Sigma x_{i}|\lambda \left( \Sigma
x_{i}\right) \leq c\right\} ,0\leq c\leq 1.$

为确定合理的$c,$考虑关%
联$R$的power function $\beta \left( p\right) =P_{p}\left( \vec{X}%
\in R\right) ,$若设定检验真实%
水平为$\alpha ,$则$c$应满足

$\underset{0<p<1}{\sup }\beta \left( p\right) =\alpha \implies P\left(
\lambda \left( \Sigma X_{i}\right) \leq c\right) =\alpha \implies $

$P\left( 2^{1000}\overline{X}^{\Sigma X_{i}}\left( 1-\overline{X}\right)
^{1000-\Sigma X_{i}}\geq \frac{1}{c}\right) =\alpha ,$取$\alpha =0.05,$%
可用随机模拟的办法%
求出$2^{1000}\overline{X}^{\Sigma X_{i}}\left( 1-\overline{X}%
\right) ^{1000-\Sigma X_{i}}$求出0.95所对应%
的分位数即为对$\frac{1}{c}$的%
估计,用R-software 可求出

$\frac{1}{c}=6.84,$表明在$c$取$\frac{1}{6.84}$%
时犯第一类错误的概%
率不超过0.05,代入题中实%
验结果

$\Sigma x_{i}=560,\lambda \left( 560\right) =0.00073<\frac{1}{6.84},$即%
$560\in R,$因此判定$H_{0}$不合理$,$%
判断错误的概率不超%
过$5\%.$

8.2 $H_{0}:$今年Poisson distribution参数$\lambda
\geq 15,H_{1}:\lambda <15,$

由于$\lambda $代表均值,$H_{1}$成%
立相当于事故率下降. 
可求出$LRT$的统计量$\mu \left(
k\right) =\frac{\underset{\lambda \geq 15}{\sup }\frac{\lambda ^{k}}{k!}%
e^{-\lambda }}{\underset{\lambda >0}{\sup }\frac{\lambda ^{k}}{k!}%
e^{-\lambda }}=\QATOPD\{ . {1,k\geq 15}{\left( \frac{15}{k}\right)
^{k}e^{k-15},k<15}$设置Rejection region by LRT method of
sufficient statistic:

$R=\left\{ k|k\leq 15\right\} =\{k<15|\left( \frac{15}{k}\right)
^{k}e^{k-15}\leq c\},0\leq c<1.$

为确定合理的$c,$考虑关%
联$R$的power function $\beta \left( \lambda \right) =P_{\lambda
}\left( X\in R\right) ,$若设定检验真%
实水平为$\alpha ,$则$c$应满足%
$\underset{15\leq \lambda }{\sup }\beta \left( \lambda \right) =\alpha
\implies \underset{15\leq \lambda }{\sup }P_{\lambda }\left( \mu \left(
X\right) \leq c\right) =\alpha ,$其中$X\symbol{126}Pois\left(
\lambda \right) $

注意到$\lambda $越大,$\mu \left( X\right) =1$%
的概率越大,因此$\underset{15\leq
\lambda }{\sup }P_{\lambda }\left( \mu \left( X\right) \leq c\right)
=P_{15}\left( \mu \left( X\right) \leq c\right) $

类似上题取$\alpha =0.05,$采用%
数值计算的方法可求%
出对$\lambda =15,$

$P_{15}\left( \mu \left( X\right) \leq c\right) =0.05,P_{15}\left( \left( 
\frac{15}{X}\right) ^{X}e^{X-15}\leq c,X<15\right) =0.05,$注意%
到

$x<15$时$f\left( x\right) =\left( \frac{15}{x}\right) ^{x}e^{x-15}$%
是单调增函数$\implies P\left(
X<f^{-1}\left( c\right) \right) =0.05;$事实上由%
于$X$

是离散型分布,不会严%
格取到0.05,因此不妨取$%
f^{-1}\left( c\right) =11,$这样使得要检%
验的观测值落入rejection region,%
可以求出对应的$\alpha =0.118.$

\bigskip 即在犯错误概率不%
超$0.118$的前提下,拒绝$H_{0}$%
是合理的.

8.3 $\Sigma Y_{i}\symbol{126}\left( m,\theta \right) $:由LME 知$%
\theta =\overline{Y}$可以极大化该样%
本的likelihood function,即LRT statistic

$\lambda \left( \Sigma y_{i}\right) =$ $\frac{\underset{\theta \leq \theta
_{0}}{\sup }\binom{m}{\Sigma y_{i}}\theta ^{\Sigma y_{i}}\left( 1-\theta
\right) ^{m-\Sigma y_{i}}}{\binom{m}{\Sigma y_{i}}\bar{y}^{\Sigma
y_{i}}\left( 1-\bar{y}\right) ^{m-\Sigma y_{i}}}=$ $\frac{\underset{\theta
\leq \theta _{0}}{\sup }\theta ^{\Sigma y_{i}}\left( 1-\theta \right)
^{m-\Sigma y_{i}}}{\bar{y}^{\Sigma y_{i}}\left( 1-\bar{y}\right) ^{m-\Sigma
y_{i}}}\QATOPD\{ . {1,\bar{y}\leq \theta _{0}}{\left( \frac{\theta _{0}}{%
\bar{y}}\right) ^{\Sigma y_{i}}\left( \frac{1-\theta _{0}}{1-\bar{y}}\right)
^{m-\Sigma y_{i}},\bar{y}>\theta _{0}},$设置Rejection region by
LRT method of sufficient statistic:$R=\left\{ \Sigma y_{i}|\lambda \left(
\Sigma y_{i}\right) \leq c\right\} 0\leq c<1$

$\implies R=\left\{ \Sigma y_{i}|\left( \frac{\theta _{0}}{\bar{y}}\right)
^{\Sigma y_{i}}\left( \frac{1-\theta _{0}}{1-\bar{y}}\right) ^{m-\Sigma
y_{i}}\leq c,\Sigma y_{i}>m\theta _{0}\right\} ,$设$x=\Sigma y_{i},$

若能判断出$\left( \frac{\theta _{0}}{\bar{y}}%
\right) ^{\Sigma y_{i}}\left( \frac{1-\theta _{0}}{1-\bar{y}}\right)
^{m-\Sigma y_{i}}=\left( \frac{m\theta _{0}}{x}\right) ^{x}\left( \frac{%
m\left( 1-\theta _{0}\right) }{m-x}\right) ^{m-x}$是关于$x$%
的单调减函数,那么立%
即可得原结论成立.为%
证此,采用取对数求导%
的办法,设$f\left( x\right) =x\ln \frac{m\theta _{0}}{%
x}+\left( m-x\right) \ln \frac{m\left( 1-\theta _{0}\right) }{m-x},f^{\prime
}\left( x\right) =\ln \frac{\theta _{0}}{1-\theta _{0}}+\ln \frac{m-x}{x}$

因$x<m\theta _{0}$而$\ln \frac{m-x}{x}$关于$x$%
单减$\implies f^{\prime }\left( x\right) <\ln \frac{\theta _{0}}{%
1-\theta _{0}}+\ln \frac{m-m\theta _{0}}{\theta _{0}}=0$

即原$\left( \frac{m\theta _{0}}{x}\right) ^{x}\left( \frac{%
m\left( 1-\theta _{0}\right) }{m-x}\right) ^{m-x}$是关于$x$%
的单调减函数$\boxtimes $

8.4由$L\left( \theta |x\right) $的定义形%
式上其为样本的联合pdf,%
于是知其表示样本观%
测值给定情况下$\theta $的%
不同取值对应的该样%
本观测值出现的概率,%
故LRT统计量的分子是观%
测样本出现的最大概%
率$\forall \theta \in \Theta _{0}$而分母是%
观测样本出现的最大%
概率$\forall \theta \in \Theta .$

8.5 $\left( 1\right) L\left( \theta ,v|x_{1},..x_{n}\right) =\frac{\theta
^{n}v^{n\theta }}{\left( \pi x_{i}\right) ^{\theta +1}}I_{[v,\infty )}\left(
x_{\left( 1\right) }\right) ,$

MLE $\hat{v}\left( \vec{x}\right) =x_{\left( 1\right) },$then maximazing $%
\theta $ is equivalent to maximaze $\theta ^{n}\left( \frac{v^{n}}{\pi x_{i}}%
\right) ^{\theta },$求导得极值$\hat{\theta}%
\left( \vec{x}\right) =\frac{n}{\ln \frac{\pi x_{i}}{x_{\left( 1\right) }^{n}%
}}.$

$\left( 2\right) $由$\left( 1\right) $的结果,可%
以很快求出关于给定%
互补的假设的LRT为$\lambda \left( 
\vec{x}\right) =\frac{\frac{\hat{v}^{n}}{\left( \pi x_{i}\right) ^{2}}}{%
\frac{\hat{\theta}^{n}\hat{v}^{n\hat{\theta}}}{\left( \pi x_{i}\right) ^{%
\hat{\theta}+1}}}=\frac{\left( \pi x_{i}\right) ^{\hat{\theta}-1}}{\hat{%
\theta}^{n}\hat{v}^{n\left( \hat{\theta}-1\right) }}=\frac{\left( \frac{\pi
x_{i}}{x_{\left( 1\right) }^{n}}\right) ^{\hat{\theta}-1}}{\hat{\theta}^{n}}%
, $rejection region:

$R\left( \vec{x}\right) =\left\{ \vec{x}|\lambda \left( \vec{x}\right) \leq
c\right\} ,\lambda \left( \vec{x}\right) \leq c,$设$T\left( \vec{x}%
\right) =\ln \frac{\pi x_{i}}{x_{\left( 1\right) }^{n}},T\left( \vec{x}%
\right) \geq 0$两边取对数有\qquad

$\left( 1-\frac{T\left( \vec{x}\right) }{n}\right) +\ln \frac{T\left( \vec{x}%
\right) }{n}\leq \frac{\ln c}{n},$由$\left( 1-x\right) +\ln x$图%
象在$x=1$取最大值0的性%
质及

$\frac{\ln c}{n}<0$可知上述不等式%
可化为$T\left( \vec{x}\right) \leq c_{1}$或$T\left( 
\vec{x}\right) \geq c_{2},$

为简便且不失一般性,%
考虑$\left( 1-x\right) +\ln x<-\mu ,\mu >0\implies x\leq b_{1}<1$
or

$x\geq b_{2}>1.b_{1},b_{2}$ satisfies the equation $\left( 1-x\right) +\ln
x=-\,\mu .$

$\implies \left( 1-b_{1}\right) +\ln b_{1}=\left( 1-b_{2}\right) +\ln
b_{2},\implies b_{2}-b_{1}=\ln \frac{b_{2}}{b_{1}}>0.$

Similarly we can get $0<c_{1}<c_{2}.$

$\left( c\right) F\left( y\right) =\int_{v}^{y}f\left( y|\theta =1,v\right)
=1-\frac{v}{y},$

考虑对于$n-1$个$X_{\left( k\right) }\left(
X_{\left( k\right) }\text{是非平凡项}\right) $

$\left( X_{k},X_{\left( 1\right) }\right) $的联合分%
布pdf $f\left( x,y\right) dxdy=\frac{v}{x^{2}}dx\frac{v}{y^{2}}%
dy\left( n-1\right) \left( 1-\frac{v}{y}\right) ^{n-2},x>y$,

做变量替换$u=\frac{x}{y},w=y\implies \left( 
\frac{X_{k}}{X_{\left( 1\right) }},X_{\left( 1\right) }\right) $的%
联合分布pdf 为$f\left( u,w\right) =\frac{%
\left( n-1\right) v^{2}}{u^{2}}\frac{1}{w^{3}}\left( 1-\frac{v}{w}\right)
^{n-2},$可见$\left( \frac{X_{k}}{X_{\left( 1\right) }},X_{\left(
1\right) }\right) $ are mutually indepedent.

$\implies \frac{X_{k}}{X_{\left( 1\right) }}$的margial distribution
pdf: $g\left( u\right) =\int_{v}^{\infty }f\left( u,w\right) dw,u>1,$积%
分(先做$t=\frac{1}{w^{2}}$的变量替%
换再利用Beta function的性质)%
可得

$g\left( u\right) =\frac{1}{u^{2}},u>1.$由于$T\left( \vec{x}%
\right) =\ln \frac{\pi x_{i}}{x_{\left( 1\right) }^{n}}$中乘%
积的效应,使得对于任%
意给定的

$X_{\left( 1\right) }=x,$总有$n-1$个非平%
凡项取值与$x$无关,即$\frac{%
\pi x_{i}}{x_{\left( 1\right) }^{n}}$可看成$n-1$个%
独立随机变量

$\frac{X_{k}}{X_{\left( 1\right) }}$的乘积$\implies 2T=%
\underset{n-1项 }{\sum }2\ln \frac{X_{k}}{X_{\left( 1\right) }},$%
通过变换不难求出$2\ln \frac{%
X_{k}}{X_{\left( 1\right) }}\symbol{126}\chi _{2}^{2},$又和%
式$n-1$

项相互独立$\implies 2T\symbol{126}\chi
_{2\left( n-1\right) }^{2}$

8.6 $\left( 1\right) L\left( \theta ,\mu \right) :=L\left( \theta ,\mu
|x_{1},..x_{m},y_{1},..y_{n}\right) =\theta ^{n}\mu ^{m}e^{-\theta \Sigma
x_{i}-\mu \Sigma y_{i}},$

\bigskip LRT $\lambda \left( x_{1},..x_{m},y_{1},..y_{n}\right) =\frac{%
\underset{\theta \neq \mu }{\sup }L\left( \theta ,\mu \right) }{\sup L\left(
\theta ,\mu \right) }$

对分子分母求极值即%
得$\lambda \left( x_{1},..x_{m},y_{1},..y_{n}\right) =\frac{\left( 
\frac{n+m}{\Sigma x_{i}+\Sigma y_{i}}\right) ^{n+m}}{\left( \frac{n}{\Sigma
x_{i}}\right) ^{n}\left( \frac{m}{\Sigma y_{i}}\right) ^{m}}.$

$\left( 2\right) $ First we can calculate $X=\Sigma X_{i}\symbol{126}\Gamma
\left( \theta ,n\right) ,Y=\Sigma Y_{i}\symbol{126}\Gamma \left( \mu
,m\right) $

$\left( X,Y\right) $ joined pdf: $\frac{\theta ^{n}\mu ^{m}}{\Gamma \left(
n\right) \Gamma \left( m\right) }x^{n-1}y^{m-1}e^{-\theta x-\mu y}dxdy$

求$T=\frac{X}{X+Y}$的分布只需做%
一个transformation $w=x,v=\frac{x}{x+y},$再对$w$%
从0到$\infty $积分,

求得$T$的pdf $f\left( v\right) =\frac{\Gamma \left(
n+m\right) }{\Gamma \left( n\right) \Gamma \left( m\right) }\theta ^{n}\mu
^{m}\frac{1}{\left( \theta +\mu \left( \frac{1}{v}-1\right) \right) ^{n+m}}%
\left( \frac{1}{v}-1\right) ^{m-1}\frac{1}{v^{2}},$由$T$产%
生的LRT

$\lambda ^{\ast }\left( \vec{x},\vec{y}\right) =\frac{v^{n+m}}{\sup \theta
^{n}\mu ^{m}\frac{1}{\left( \theta +\mu \left( \frac{1}{v}-1\right) \right)
^{n+m}}},$对分母求最值有

$=\frac{v^{n+m}}{\frac{m^{m}n^{n}}{\left( \frac{1}{v}-1\right) ^{m}\left(
m+n\right) ^{m+n}}}$化简得$=\frac{\left( \frac{n+m}{\Sigma
x_{i}+\Sigma y_{i}}\right) ^{n+m}}{\left( \frac{n}{\Sigma x_{i}}\right)
^{n}\left( \frac{m}{\Sigma y_{i}}\right) ^{m}},$与$\left( 1\right) $%
中所求相同$.$

$\left( 3\right) \theta =\mu \implies f\left( v\right) =\frac{\Gamma \left(
n+m\right) }{\Gamma \left( n\right) \Gamma \left( m\right) }v^{n-1}\left(
1-v\right) ^{m-1}\implies T\symbol{126}Beta\left( n,m\right) $

8.7 $\left( 1\right) $LRT$\qquad L\left( \theta ,\lambda
|x_{1},..x_{n}\right) =\frac{1}{\lambda }e^{\frac{-\left( \Sigma
x_{i}-n\theta \right) }{\lambda }}I_{[\theta ,\infty )}\left( x_{\left(
1\right) }\right) $

$\lambda \left( x_{1},..x_{n}\right) =\frac{\underset{\theta \leq 0}{\sup }%
L\left( \theta ,\mu \right) }{\sup L\left( \theta ,\mu \right) }=\QATOPD\{ .
{1,x_{\left( 1\right) }\leq 0}{\frac{\Sigma x_{i}-nx_{\left( 1\right) }}{%
\Sigma x_{i}}x_{\left( 1\right) }>0}$

$\left( 2\right) L\left( \gamma ,\beta |x_{1},..x_{n}\right) =\frac{\gamma
^{n}}{\beta ^{n}}\left( \pi x_{i}\right) ^{\gamma -1}e^{-\frac{\Sigma
x_{i}^{\gamma }}{\beta }}$

$\lambda \left( x_{1},..x_{n}\right) =\frac{\underset{\gamma =1}{\sup }%
L\left( \gamma ,\beta \right) }{\sup L\left( \gamma ,\beta \right) }=\frac{%
\left( \frac{n}{\Sigma x_{i}}\right) ^{n}e^{-n}}{\sup L\left( \gamma ,\beta
\right) },$

Notice that if the estimation of $\gamma $ is known, then by diffierential
method $\hat{\beta}\left( \vec{x}\right) =\frac{\Sigma x_{i}^{\gamma }}{n}.$%
However, $\gamma $ satisfies an implicit function: $\frac{n}{\gamma }+\ln
\left( \pi x_{i}\right) -\frac{n}{\Sigma x_{i}^{\gamma }}\Sigma \ln
x_{i}\left( x_{i}^{\gamma }\right) =0,$which should be solved numerically.

8.8 $\left( 1\right) $考虑 sufficient statistic $\overline{X}%
\symbol{126}N\left( \theta ,\frac{a\theta }{n}\right) ,L\left( a,\theta |%
\bar{x}\right) =\frac{1}{\sqrt{2\pi }\sqrt{\frac{a\theta }{n}}}e^{-\frac{%
\left( \bar{x}-\theta \right) ^{2}}{2\frac{a\theta }{n}}}$

$\lambda \left( x_{1},..x_{n}\right) =\lambda ^{\ast }\left( \bar{x}\right) =%
\frac{\underset{a=1}{\sup }L\left( a,\theta |\bar{x}\right) }{\sup L\left(
a,\theta |\bar{x}\right) },$求$\sup L\left( a,\theta |\bar{x}\right) $%
相当于求$\inf \sqrt{\theta }e^{\frac{\left( \bar{x}%
-\theta \right) ^{2}}{2\frac{a\theta }{n}}}$

求导法给出最小值点$%
\theta =\frac{-\frac{a}{n}+\sqrt{\frac{a^{2}}{n^{2}}+4\bar{x}^{2}}}{2},$%
代入$\lambda ^{\ast }\left( \bar{x}\right) $分母,%
再取$a=1$代入$\lambda ^{\ast }\left( \bar{x}\right) $%
分子,

即得LRT: $\left\{ \bar{x}|\lambda ^{\ast }\left( \bar{x}\right)
<c\right\} .$

$\left( 2\right) $考虑 sufficient statistic $\overline{X}\symbol{%
126}N\left( \theta ,\frac{a\theta ^{2}}{n}\right) ,L\left( a,\theta |\bar{x}%
\right) =\frac{1}{\sqrt{2\pi }\sqrt{\frac{a}{n}}\theta }e^{-\frac{\left( 
\bar{x}-\theta \right) ^{2}}{2\frac{a\theta ^{2}}{n}}}$

$\lambda \left( x_{1},..x_{n}\right) =\lambda ^{\ast }\left( \bar{x}\right) =%
\frac{\underset{a=1}{\sup }L\left( a,\theta |\bar{x}\right) }{\sup L\left(
a,\theta |\bar{x}\right) },$求$\sup L\left( a,\theta |\bar{x}\right) $%
相当于求$\inf \theta e^{\frac{\left( \bar{x}-\theta
\right) ^{2}}{2\frac{a\theta ^{2}}{n}}}$

求导法给出最小值点$%
\theta =\frac{-\frac{n}{a}+\sqrt{\frac{n^{2}}{a^{2}}+4\frac{n}{a}}}{2}\bar{x}%
,$代入$\lambda ^{\ast }\left( \bar{x}\right) $分母,%
再取$a=1$代入$\lambda ^{\ast }\left( \bar{x}\right) $%
分子,

即得LRT: $\left\{ \bar{x}|\lambda ^{\ast }\left( \bar{x}\right)
<c\right\} .$

\end{document}
