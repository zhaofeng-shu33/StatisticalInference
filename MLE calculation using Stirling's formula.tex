
\documentclass{article}
\usepackage{amsmath}

%%%%%%%%%%%%%%%%%%%%%%%%%%%%%%%%%%%%%%%%%%%%%%%%%%%%%%%%%%%%%%%%%%%%%%%%%%%%%%%%%%%%%%%%%%%%%%%%%%%%%%%%%%%%%%%%%%%%%%%%%%%%%%%%%%%%%%%%%%%%%%%%%%%%%%%%%%%%%%%%%%%%%%%%%%%%%%%%%%%%%%%%%%%%%%%%%%%%%%%%%%%%%%%%%%%%%%%%%%%%%%%%%%%%
%TCIDATA{OutputFilter=LATEX.DLL}
%TCIDATA{Version=5.00.0.2552}
%TCIDATA{<META NAME="SaveForMode" CONTENT="1">}
%TCIDATA{Created=Tuesday, November 03, 2015 19:08:41}
%TCIDATA{LastRevised=Tuesday, November 03, 2015 19:10:08}
%TCIDATA{<META NAME="GraphicsSave" CONTENT="32">}
%TCIDATA{<META NAME="DocumentShell" CONTENT="Scientific Notebook\Blank Document">}
%TCIDATA{CSTFile=Math with theorems suppressed.cst}
%TCIDATA{PageSetup=72,72,72,72,0}
%TCIDATA{AllPages=
%F=36,\PARA{038<p type="texpara" tag="Body Text" >\hfill \thepage}
%}


\newtheorem{theorem}{Theorem}
\newtheorem{acknowledgement}[theorem]{Acknowledgement}
\newtheorem{algorithm}[theorem]{Algorithm}
\newtheorem{axiom}[theorem]{Axiom}
\newtheorem{case}[theorem]{Case}
\newtheorem{claim}[theorem]{Claim}
\newtheorem{conclusion}[theorem]{Conclusion}
\newtheorem{condition}[theorem]{Condition}
\newtheorem{conjecture}[theorem]{Conjecture}
\newtheorem{corollary}[theorem]{Corollary}
\newtheorem{criterion}[theorem]{Criterion}
\newtheorem{definition}[theorem]{Definition}
\newtheorem{example}[theorem]{Example}
\newtheorem{exercise}[theorem]{Exercise}
\newtheorem{lemma}[theorem]{Lemma}
\newtheorem{notation}[theorem]{Notation}
\newtheorem{problem}[theorem]{Problem}
\newtheorem{proposition}[theorem]{Proposition}
\newtheorem{remark}[theorem]{Remark}
\newtheorem{solution}[theorem]{Solution}
\newtheorem{summary}[theorem]{Summary}
\newenvironment{proof}[1][Proof]{\noindent\textbf{#1.} }{\ \rule{0.5em}{0.5em}}


\begin{document}


\bigskip Binomial Example with unknown number of trial.

Estimating the number of trial with MLE.

Textbook P318

Here is another approach to derive the same equation, when $k>>n.$

$\ln \left( L\right) =n\ln k!-\underset{i=1}{\overset{n}{\sum }}\left(
k-x_{i}\right) !+\left( nk-\underset{i=1}{\overset{n}{\sum }}x_{i}\right)
\ln \left( 1-p\right) +c,$with $c$ is irrelevant with $k.$

Using Stirling's formula $\ln x!\approx x\ln \left( x\right) -x$

$\implies \ln \left( L\right) =nk\ln \left( k\right) -nk-\underset{i=1}{%
\overset{n}{\sum }}\left( k-x_{i}\right) \ln \left( k-x_{i}\right) +\underset%
{i=1}{\overset{n}{\sum }}\left( k-x_{i}\right) +\left( nk-\underset{i=1}{%
\overset{n}{\sum }}x_{i}\right) \ln \left( 1-p\right) +c$

=$nk\ln \left( k\right) -\underset{i=1}{\overset{n}{\sum }}\left(
k-x_{i}\right) \ln \left( k-x_{i}\right) +\left( nk-\underset{i=1}{\overset{n%
}{\sum }}x_{i}\right) \ln \left( 1-p\right) +c^{\prime }$

Taking the derivative of $\ln \left( L\right) $ about $k$ gives:

$\frac{d\ln \left( L\right) }{dk}=n\ln k+n-n-\underset{i=1}{\overset{n}{\sum 
}}\ln \left( k-x_{i}\right) +n\ln \left( 1-p\right) $

=$n\ln k-\underset{i=1}{\overset{n}{\sum }}\ln \left( k-x_{i}\right) +n\ln
\left( 1-p\right) $

Let $\frac{d\ln \left( L\right) }{dk}=0$ follows $\left( 1-p\right) ^{n}=%
\underset{i=1}{\overset{n}{\Pi }}\left( 1-\frac{x_{i}}{k}\right) ,$ which is
the same with that solved in textbook.

\end{document}
