
\documentclass{article}
%%%%%%%%%%%%%%%%%%%%%%%%%%%%%%%%%%%%%%%%%%%%%%%%%%%%%%%%%%%%%%%%%%%%%%%%%%%%%%%%%%%%%%%%%%%%%%%%%%%%%%%%%%%%%%%%%%%%%%%%%%%%%%%%%%%%%%%%%%%%%%%%%%%%%%%%%%%%%%%%%%%%%%%%%%%%%%%%%%%%%%%%%%%%%%%%%%%%%%%%%%%%%%%%%%%%%%%%%%%%%%%%%%%%%%%%%%%%%%%%%%%%%%%%%%%%
\usepackage{amsmath}
\def\QATOPD#1#2#3#4{{#3 \atopwithdelims#1#2 #4}}%

\setcounter{MaxMatrixCols}{10}
%TCIDATA{OutputFilter=LATEX.DLL}
%TCIDATA{Version=5.00.0.2552}
%TCIDATA{<META NAME="SaveForMode" CONTENT="1">}
%TCIDATA{Created=Sunday, November 15, 2015 22:24:49}
%TCIDATA{LastRevised=Wednesday, November 18, 2015 01:04:40}
%TCIDATA{<META NAME="GraphicsSave" CONTENT="32">}
%TCIDATA{<META NAME="DocumentShell" CONTENT="Scientific Notebook\Blank Document">}
%TCIDATA{CSTFile=Math with theorems suppressed.cst}
%TCIDATA{PageSetup=72,72,72,72,0}
%TCIDATA{AllPages=
%F=36,\PARA{038<p type="texpara" tag="Body Text" >\hfill \thepage}
%}


\newtheorem{theorem}{Theorem}
\newtheorem{acknowledgement}[theorem]{Acknowledgement}
\newtheorem{algorithm}[theorem]{Algorithm}
\newtheorem{axiom}[theorem]{Axiom}
\newtheorem{case}[theorem]{Case}
\newtheorem{claim}[theorem]{Claim}
\newtheorem{conclusion}[theorem]{Conclusion}
\newtheorem{condition}[theorem]{Condition}
\newtheorem{conjecture}[theorem]{Conjecture}
\newtheorem{corollary}[theorem]{Corollary}
\newtheorem{criterion}[theorem]{Criterion}
\newtheorem{definition}[theorem]{Definition}
\newtheorem{example}[theorem]{Example}
\newtheorem{exercise}[theorem]{Exercise}
\newtheorem{lemma}[theorem]{Lemma}
\newtheorem{notation}[theorem]{Notation}
\newtheorem{problem}[theorem]{Problem}
\newtheorem{proposition}[theorem]{Proposition}
\newtheorem{remark}[theorem]{Remark}
\newtheorem{solution}[theorem]{Solution}
\newtheorem{summary}[theorem]{Summary}
\newenvironment{proof}[1][Proof]{\noindent\textbf{#1.} }{\ \rule{0.5em}{0.5em}}


\begin{document}


$\bigskip 7.54\left( a\right) $the information contained in T is E$\left( 
\frac{\partial }{\partial \theta }\log p\left( x|\theta \right) \right) ^{2}.
$

$\left\vert \frac{\partial }{\partial \theta }\log p\left( x|\theta \right)
\right\vert =2n\frac{\theta ^{2}-x^{2}}{\theta \left( \theta
^{2}+x^{2}\right) }.$E$\left( \frac{\partial }{\partial \theta }\log p\left(
x|\theta \right) \right) ^{2}$

$=4n^{2}\frac{\Gamma \left( 2n\right) }{\Gamma \left( n\right) ^{2}}%
\int_{0}^{\infty }\left( \frac{\theta ^{2}-x^{2}}{\theta \left( \theta
^{2}+x^{2}\right) }\right) ^{2}\frac{2x^{2n-1}}{(\theta +\frac{x^{2}}{\theta 
})^{2n}}dx=\frac{4n^{2}}{2n+1}\frac{1}{\theta ^{2}}($use Mathematica to
compute)

=$\frac{2n}{2n+1}I\left( \theta \right) .$

For $\left( X_{i},Y_{i}\right) ,i.i.d$ from the population, the log
likelihood function is

$\log L=-\Sigma \left( \theta x_{i}+\frac{y_{i}}{\theta }\right) ,$ $\frac{%
\partial \log L}{\partial \theta }=0\implies \hat{\theta}=\sqrt{\frac{\Sigma
y_{i}}{\Sigma x_{i}}}$

From 6.37 T is defined as $T=\sqrt{\frac{\Sigma Y_{i}}{\Sigma X_{i}}}%
\implies T$ is the MLE.

$\Sigma X_{i}\symbol{126}\Gamma \left( \theta ,n\right) ,\Sigma Y_{i}\symbol{%
126}\Gamma \left( \frac{1}{\theta },n\right) .$Making transformation $u=%
\sqrt{xy},v=y$ gives

the joint pdf of $\left( U,V\right) :p\left( u,v\right) =\frac{2u}{v}\frac{%
u^{2\left( n-1\right) }}{\Gamma \left( n\right) ^{2}}e^{-\left( \theta \frac{%
u^{2}}{v}+\frac{v}{\theta }\right) }.$Integrating about v gives the pdf of
U: $p\left( u\right) =\frac{2u^{2n-1}}{\Gamma \left( n\right) ^{2}}%
\int_{0}^{\infty }\frac{1}{v}e^{-\left( \theta \frac{u^{2}}{v}+\frac{v}{%
\theta }\right) }dv=\frac{2u^{2n-1}}{\Gamma \left( n\right) ^{2}}%
\int_{0}^{\infty }\frac{1}{v}e^{-\left( \frac{u^{2}}{v}+v\right) }dv$

which is irrelevant with $\theta \implies U$ is an ancillary statistic.

\bigskip Similarly, we can calculate the pdf of T: p(x)=$\frac{\Gamma \left(
2n\right) }{\Gamma \left( n\right) ^{2}}\frac{2x^{2n-1}}{(\theta +\frac{x^{2}%
}{\theta })^{2n}},$

$\bigskip $Using Mathematica to compute gives

$E\left( T\right) =\frac{\Gamma \left( 2n\right) }{\Gamma \left( n\right)
^{2}}\int_{0}^{\infty }\frac{2x^{2n}}{(\theta +\frac{x^{2}}{\theta })^{2n}}%
dx=\frac{\Gamma \left( 2n\right) }{\Gamma \left( n\right) ^{2}}\theta
\int_{0}^{\infty }\frac{2x^{2n}}{(1+x^{2})^{2n}}dx=\frac{2^{1-2n}\sqrt{\pi }%
\Gamma \left( 2n\right) \Gamma \left( n-\frac{1}{2}\right) }{\Gamma \left(
n\right) ^{3}}$

=$\frac{\left( 2n-1\right) !}{\left( n-1\right) !2^{2n-1}}\Gamma \left( 
\frac{1}{2}\right) \frac{\Gamma \left( n-\frac{1}{2}\right) }{\Gamma \left(
n\right) ^{2}}.$Since $\left( 2n-1\right) !=\left( 2n-1\right) !!\left(
2n-2\right) !!$

=$\left( 2n-1\right) !!\left( n-1\right) !2^{n-1}\implies E\left( T\right) =%
\frac{\left( 2n-1\right) !!}{2^{n}}\Gamma \left( \frac{1}{2}\right) \frac{%
\Gamma \left( n-\frac{1}{2}\right) }{\Gamma \left( n\right) ^{2}}$

$=\left( n-\frac{1}{2}\right) ..\left( \frac{1}{2}\right) \Gamma \left( 
\frac{1}{2}\right) \frac{\Gamma \left( n-\frac{1}{2}\right) }{\Gamma \left(
n\right) ^{2}}=\frac{\Gamma \left( n+\frac{1}{2}\right) \Gamma \left( n-%
\frac{1}{2}\right) }{\Gamma \left( n\right) ^{2}}.$

$E\left( T^{2}\right) =\frac{\Gamma \left( 2n\right) }{\Gamma \left(
n\right) ^{2}}\theta ^{2}\int_{0}^{\infty }\frac{2x^{2n+1}}{(1+x^{2})^{2n}}%
dx=\theta ^{2}\frac{\Gamma \left( n+1\right) \Gamma \left( n-1\right) }{%
\Gamma \left( n\right) ^{2}}.$

$\Sigma X_{i}\symbol{126}\Gamma \left( \theta ,n\right)
,EZ_{1}=\int_{0}^{\infty }\frac{n-1}{x}\frac{\theta ^{n}}{\Gamma \left(
n\right) }x^{n-1}e^{-\theta x}dx=\theta .$

$Y_{i}\symbol{126}Exp\left( \frac{1}{\theta }\right) \implies
EZ_{2}=EY_{1}=\theta .$

Therefore both $Z_{1}$ and $Z_{2}$ are unbiased estimator of $\theta .$

$VarZ_{1}=EZ_{1}^{2}-E^{2}Z_{1}=\int_{0}^{\infty }\frac{\left( n-1\right)
^{2}}{x^{2}}\frac{\theta ^{n}}{\Gamma \left( n\right) }x^{n-1}e^{-\theta
x}dx-\theta ^{2}=\frac{1}{n-2}\theta ^{2}.$

$VarZ_{2}=\frac{VarY_{1}}{n}=\frac{\theta ^{2}}{n}.$

$\left( c\right) $ Since $X_{i}$ is indepedent with $Y_{i}$

$\implies Var\left( aZ_{1}+\left( 1-a\right) Z_{2}\right)
=a^{2}VarZ_{1}+\left( 1-a\right) ^{2}VarZ_{2}$

Minimizing $a^{2}VarZ_{1}+\left( 1-a\right) ^{2}VarZ_{2}$ gives $a=\frac{%
VarZ_{2}}{VarZ_{1}+VarZ_{2}}=\frac{n-2}{2n-2}.$

Therefore the best unbiased estimator of the form $aZ_{1}+\left( 1-a\right)
Z_{2}$

is $\frac{n-2}{2n-2}Z_{1}+\frac{n}{2n-2}Z_{2}$ and its variance is $\frac{1}{%
2n-2}\theta .$

The bias-corrected MLE is $\frac{\Gamma \left( n\right) ^{2}}{\Gamma \left(
n+\frac{1}{2}\right) \Gamma \left( n-\frac{1}{2}\right) }T,$and its variance
follows from $\left( a\right) $

is $\left( \frac{\Gamma \left( n\right) ^{2}}{\Gamma \left( n+\frac{1}{2}%
\right) \Gamma \left( n-\frac{1}{2}\right) }\right) ^{2}\left( \frac{\Gamma
\left( n+1\right) \Gamma \left( n-1\right) }{\Gamma \left( n\right) ^{2}}%
-\left( \frac{\Gamma \left( n+\frac{1}{2}\right) \Gamma \left( n-\frac{1}{2}%
\right) }{\Gamma \left( n\right) ^{2}}\right) ^{2}\right) \theta ^{2}$

$=\left( \frac{\Gamma \left( n\right) ^{2}\Gamma \left( n+1\right) \Gamma
\left( n-1\right) }{\Gamma ^{2}\left( n+\frac{1}{2}\right) \Gamma ^{2}\left(
n-\frac{1}{2}\right) }-1\right) \theta ^{2}=\left( \frac{\left( n-1\right)
!^{2}n!\left( n-2\right) !}{\left( \frac{\left( 2n-1\right) !!}{2^{n}}\frac{%
\left( 2n-3\right) !!}{2^{n-1}}\right) ^{2}\pi }-1\right) \theta ^{2}$

$\frac{\left( 2n!!\right) }{\sqrt{n}\left( 2n-1\right) !!}\frac{\left(
2n-2\right) !!^{2}}{n\left( 2n-3\right) !!^{2}}\frac{\left( 2n-4\right) !!}{%
\sqrt{n}\left( 2n-5\right) !!}\frac{2^{2}n^{2}}{\left( 2n-1\right) \left(
2n-3\right) \pi }.$

By Wallis Formular, $\frac{\left( 2n!!\right) }{\sqrt{n}\left( 2n-1\right) !!%
}\rightarrow \sqrt{\pi }\implies \frac{\left( n-1\right) !^{2}n!\left(
n-2\right) !}{\left( \frac{\left( 2n-1\right) !!}{2^{n}}\frac{\left(
2n-3\right) !!}{2^{n-1}}\right) ^{2}\pi }\rightarrow \frac{4\pi n^{2}}{%
\left( 2n-1\right) \left( 2n-3\right) }$

As n is relatively large, the calculated variance is $\left( \frac{4\pi n^{2}%
}{\left( 2n-1\right) \left( 2n-3\right) }-1\right) \theta \approx \left( \pi
-1\right) \theta ,$which is much large than $\frac{1}{2n-2}\theta .$

7-57 $\left( a\right) $ $\underset{i=1}{\overset{n}{\Sigma }}X_{i}\symbol{126%
}B\left( n,p\right) ,$and since $\underset{i=1}{\overset{n}{\Sigma }}X_{i}$
is indepedent with $X_{n+1}\implies $ $h\left( p\right) =P\left( \underset{%
i=1}{\overset{n}{\Sigma }}X_{i}>X_{n+1}|p\right) $

$=\left( 1-p\right) \left( 1-\left( 1-p\right) ^{n}\right) +p\left( 1-\left(
1-p\right) ^{n}-n\left( 1-p\right) ^{n-1}p\right) $

$=\allowbreak 1-\left( 1-p\right) ^{n}-np^{2}\left( 1-p\right) ^{n-1}.$

$E\left( T\right) =P\left( \underset{i=1}{\overset{n}{\Sigma }}%
X_{i}>X_{n+1}|p\right) =h\left( p\right) .$ From the former chaper we know

$\underset{i=1}{\overset{n+1}{\Sigma }}X_{i}$ is a complete (about p)
sufficient statistic of the sample.

\bigskip By Rao-Blackwell Thm, $E\left( T|\underset{i=1}{\overset{n+1}{%
\Sigma }}X_{i}\right) $ is the unique best unbiased estimator of $h\left(
p\right) .$

\bigskip $E\left( T|\underset{i=1}{\overset{n+1}{\Sigma }}X_{i}=t\right)
=P\left( \underset{i=1}{\overset{n}{\Sigma }}X_{i}>X_{n+1}|\underset{i=1}{%
\overset{n+1}{\Sigma }}X_{i}=t\right) $

$=\frac{P\left( \underset{i=1}{\overset{n}{\Sigma }}X_{i}>X_{n+1},\underset{%
i=1}{\overset{n+1}{\Sigma }}X_{i}=t\right) }{P\left( \underset{i=1}{\overset{%
n+1}{\Sigma }}X_{i}=t\right) }=\frac{P\left( \underset{i=1}{\overset{n}{%
\Sigma }}X_{i}>1,X_{n+1}=1,\underset{i=1}{\overset{n}{\Sigma }}%
X_{i}=t-1\right) +P\left( \underset{i=1}{\overset{n}{\Sigma }}%
X_{i}>0,X_{n+1}=0,\underset{i=1}{\overset{n}{\Sigma }}X_{i}=t\right) }{%
P\left( \underset{i=1}{\overset{n+1}{\Sigma }}X_{i}=t\right) }$

if $t\geq 3,$the above equality=$\frac{1}{\binom{n+1}{t}p^{t}\left(
1-p\right) ^{n+1-t}};$

if $t=2$ or 1$,$the above equality=$\frac{\left( 1-p\right) }{\binom{n+1}{t}%
p^{t}\left( 1-p\right) ^{n+1-t}};$

if $t=0,$the above equality=0.

$\implies E\left( T|\underset{i=1}{\overset{n+1}{\Sigma }}X_{i}\right)
=\QATOPD\{ . {\frac{1}{\binom{n+1}{\Sigma X_{i}}p^{\Sigma X_{i}}\left(
1-p\right) ^{n+1-\Sigma X_{i}}},\Sigma X_{i}\geq 3}{\underset{0,\text{ \ \ \
\ }\Sigma X_{i}=0}{\frac{1-p}{\binom{n+1}{\Sigma X_{i}}p^{\Sigma
X_{i}}\left( 1-p\right) ^{n+1-\Sigma X_{i}}},\Sigma X_{i}=2\text{ or 1}}}.$

7-58. $\left( a\right) $ The log likelyhood function $\log f=\left\vert
x\right\vert \log \frac{\theta }{2}+\left( 1-\left\vert x\right\vert \right)
\log \left( 1-\theta \right) $

$\bigskip \frac{\partial \log f}{\partial \theta }=0\implies \frac{%
\left\vert x\right\vert }{\theta }-\frac{1-\left\vert x\right\vert }{%
1-\theta }=0\implies MLE:\hat{\theta}=\left\vert x\right\vert .$

Further verification:If $x=0,$the max of f is reached at $\theta =0,$

if $\left\vert x\right\vert =1,$the max of f is reached at $\theta =1,$

$\left( b\right) $ $E\left( T\right) =2\frac{\theta }{2}=\theta $

$\implies T$ is an unbiased estimator of $\theta .$

$\left( c\right) Var\left( T\right) =E\left( T^{2}\right) -E^{2}T=2\theta
-\theta ^{2},$

Let $S\left( X\right) =\QATOPD\{ . {1,if\text{ }x=1}{1,if\text{ }%
x=-1}.E\left( S\right) =\theta ,\implies S\left( X\right) $ is an unbiased
estimator of $\theta .$

$Var\left( S\right) =E\left( S^{2}\right) -E^{2}T=\theta -\theta
^{2}<Var\left( T\right) \implies $

$S\left( X\right) $ is a better unbiased estimator of $\theta .$

\end{document}
