
\documentclass{article}
%%%%%%%%%%%%%%%%%%%%%%%%%%%%%%%%%%%%%%%%%%%%%%%%%%%%%%%%%%%%%%%%%%%%%%%%%%%%%%%%%%%%%%%%%%%%%%%%%%%%%%%%%%%%%%%%%%%%%%%%%%%%%%%%%%%%%%%%%%%%%%%%%%%%%%%%%%%%%%%%%%%%%%%%%%%%%%%%%%%%%%%%%%%%%%%%%%%%%%%%%%%%%%%%%%%%%%%%%%%%%%%%%%%%%%%%%%%%%%%%%%%%%%%%%%%%
\usepackage{amsmath}

\setcounter{MaxMatrixCols}{10}
%TCIDATA{OutputFilter=LATEX.DLL}
%TCIDATA{Version=5.00.0.2552}
%TCIDATA{<META NAME="SaveForMode" CONTENT="1">}
%TCIDATA{Created=Wednesday, December 09, 2015 18:26:45}
%TCIDATA{LastRevised=Sunday, December 13, 2015 16:00:06}
%TCIDATA{<META NAME="GraphicsSave" CONTENT="32">}
%TCIDATA{<META NAME="DocumentShell" CONTENT="Standard LaTeX\Blank - Standard LaTeX Article">}
%TCIDATA{CSTFile=40 LaTeX article.cst}

\newtheorem{theorem}{Theorem}
\newtheorem{acknowledgement}[theorem]{Acknowledgement}
\newtheorem{algorithm}[theorem]{Algorithm}
\newtheorem{axiom}[theorem]{Axiom}
\newtheorem{case}[theorem]{Case}
\newtheorem{claim}[theorem]{Claim}
\newtheorem{conclusion}[theorem]{Conclusion}
\newtheorem{condition}[theorem]{Condition}
\newtheorem{conjecture}[theorem]{Conjecture}
\newtheorem{corollary}[theorem]{Corollary}
\newtheorem{criterion}[theorem]{Criterion}
\newtheorem{definition}[theorem]{Definition}
\newtheorem{example}[theorem]{Example}
\newtheorem{exercise}[theorem]{Exercise}
\newtheorem{lemma}[theorem]{Lemma}
\newtheorem{notation}[theorem]{Notation}
\newtheorem{problem}[theorem]{Problem}
\newtheorem{proposition}[theorem]{Proposition}
\newtheorem{remark}[theorem]{Remark}
\newtheorem{solution}[theorem]{Solution}
\newtheorem{summary}[theorem]{Summary}
\newenvironment{proof}[1][Proof]{\noindent\textbf{#1.} }{\ \rule{0.5em}{0.5em}}


\begin{document}


Problem 8.21:

Before proof, we revise the statement of Thm 8.3.12 slightly.

$R=\{x|f\left( x|\theta _{1}\right) \geq kf\left( x|\theta _{0}\right) \}.$

The testing $H_{0}:\theta =\theta _{0}$ versus $H_{1}:\theta =\theta _{1},$%
where the pmf corresponding to

$\theta _{i}$ is $f\left( x|\theta _{i}\right) ,X$ is a discrete random
variable. Below we show Neyman Pearson lemma for such case.

\bigskip Sufficiency: Suppose a test $\phi $ satisfies $\left( 8.3.1\right) $
and $\left( 8.3.2\right) $

$\underset{\theta \in \Theta }{\sup }P_{\theta }\left( X\in R\right)
=P_{\theta _{0}}\left( X\in R\right) =\alpha ,\implies \phi $ is a level $%
\alpha $ test. For any other test $\phi ^{\prime }$ satisfying $P_{\theta
_{0}}(X\in R^{\prime })\leq \alpha ,$ we shall show that $P_{\theta
_{1}}\left( X\in R\right) \geq P_{\theta _{1}}\left( X\in R^{\prime }\right)
.$

$P_{\theta _{1}}\left( X\in R\right) =P_{\theta _{1}}\left( X\in R\right)
-k\left( P_{\theta _{0}}\left( X\in R\right) -\alpha \right) $

$=P_{\theta _{1}}\left( X\in R\right) -kP_{\theta _{0}}\left( X\in R\right)
+k\alpha $

Since $X$ is a discrete random variable, the sample space $\Omega $ has at
most countable sample points$\implies P_{\theta _{1}}\left( X\in R\right) =%
\underset{x_{i}\in R}{\sum }P_{\theta _{1}}\left( X=x_{i}\right) =\underset{%
x_{i}\in R}{\sum }f\left( x_{i}|\theta _{1}\right) \implies $

$P_{\theta _{1}}\left( X\in R\right) =\underset{x_{i}\in R}{\sum }\left(
f\left( x_{i}|\theta _{1}\right) -kf\left( x_{i}|\theta _{0}\right) \right)
+k\alpha .$

$\implies P_{\theta _{1}}\left( X\in R\right) \geq \underset{x_{i}\in R}{%
\sum }\left( f\left( x_{i}|\theta _{1}\right) -kf\left( x_{i}|\theta
_{0}\right) \right) +kP_{\theta _{0}}(X\in R^{\prime })$

$=\underset{x_{i}\in R}{\sum }\left( f\left( x_{i}|\theta _{1}\right)
-kf\left( x_{i}|\theta _{0}\right) \right) +k\underset{x_{i}\in R^{\prime }}{%
\sum }f\left( x_{i}|\theta _{0}\right) $

$=\underset{x_{i}\in R\cap R^{\prime c}}{\sum }\left( f\left( x_{i}|\theta
_{1}\right) -kf\left( x_{i}|\theta _{0}\right) \right) +\underset{x_{i}\in
R\cap R^{\prime }}{\sum }\left( f\left( x_{i}|\theta _{1}\right) -kf\left(
x_{i}|\theta _{0}\right) \right) +k\underset{x_{i}\in R^{\prime }}{\sum }%
f\left( x_{i}|\theta _{0}\right) $

$=\underset{x_{i}\in R\cap R^{\prime c}}{\sum }\left( f\left( x_{i}|\theta
_{1}\right) -kf\left( x_{i}|\theta _{0}\right) \right) +\underset{x_{i}\in
R\cap R^{\prime }}{\sum }f\left( x_{i}|\theta _{1}\right) +k\left( \underset{%
x_{i}\in R^{\prime }}{\sum }f\left( x_{i}|\theta _{0}\right) -\underset{%
x_{i}\in R\cap R^{\prime }}{\sum }f\left( x_{i}|\theta _{0}\right) \right) $

$\geq \underset{x_{i}\in R\cap R^{\prime }}{\sum }f\left( x_{i}|\theta
_{1}\right) +k\underset{x_{i}\in R^{\prime }\cap R^{c}}{\sum }f\left(
x_{i}|\theta _{0}\right) \left( \text{we drop the first term,which is
positive}\right) $

$P_{\theta _{1}}\left( X\in R^{\prime }\right) =\underset{x_{i}\in R^{\prime
}}{\sum }f\left( x_{i}|\theta _{1}\right) =\underset{x_{i}\in R^{\prime
}\cap R}{\sum }f\left( x_{i}|\theta _{1}\right) +\underset{x_{i}\in
R^{\prime }\cap R^{c}}{\sum }f\left( x_{i}|\theta _{1}\right) $

$\leq \underset{x_{i}\in R^{\prime }\cap R}{\sum }f\left( x_{i}|\theta
_{1}\right) +\underset{x_{i}\in R^{\prime }\cap R^{c}}{\sum }kf\left(
x_{i}|\theta _{0}\right) $

Since $f\left( x_{i}|\theta _{1}\right) -kf\left( x_{i}|\theta _{0}\right)
<0,$as $x_{i}\in R^{c}$

$\implies P_{\theta _{1}}\left( X\in R\right) \geq P_{\theta _{1}}\left(
X\in R^{\prime }\right) $

$\implies \phi $ is a UMP level $\alpha $ test.

Necessity: Suppose a test $\phi \left( R\right) $ satisfies $\left(
8.3.1\right) $ and $\left( 8.3.2\right) $ with $k>0,$ then

for each UMP level $\alpha $ test $\phi ^{\prime }\left( R^{\prime }\right)
, $ we have $P_{\theta _{1}}\left( X\in R^{\prime }\right) \geq P_{\theta
_{1}}\left( X\in R\right) ,$since

$\phi ^{\prime }$ is UMP$.$

By the proof of sufficiency, we know that $\phi $ is also a UMP level $%
\alpha $ test

$\implies P_{\theta _{1}}\left( X\in R^{\prime }\right) \leq P_{\theta
_{1}}\left( X\in R\right) \implies P_{\theta _{1}}\left( X\in R^{\prime
}\right) =P_{\theta _{1}}\left( X\in R\right) .$

To show $P_{\theta _{0}}\left( X\in R^{\prime }\right) =\alpha ,$ we already
know that $P_{\theta _{0}}\left( X\in R^{\prime }\right) \leq \alpha $

Notice that in the proof of sufficiency, we already get a sequence of
inequality:

$\bigskip P_{\theta _{1}}\left( X\in R\right) =\underset{x_{i}\in R}{\sum }%
\left( f\left( x_{i}|\theta _{1}\right) -kf\left( x_{i}|\theta _{0}\right)
\right) +k\alpha $

$\geq \underset{x_{i}\in R}{\sum }\left( f\left( x_{i}|\theta _{1}\right)
-kf\left( x_{i}|\theta _{0}\right) \right) +kP_{\theta _{0}}(X\in R^{\prime
})...\geq P_{\theta _{1}}\left( X\in R^{\prime }\right) $

The equality must be reached$\implies \alpha =$ $\bigskip P_{\theta
_{0}}(X\in R^{\prime }),$

$\bigskip $Further from the condition of equality above we have:

$f\left( x_{i}|\theta _{1}\right) =kf\left( x_{i}|\theta _{0}\right)
,\forall x_{i}\in R\cap R^{\prime c},R^{\prime }\cap R^{c}.$

$f\left( x_{i}|\theta _{1}\right) <kf\left( x_{i}|\theta _{0}\right)
,\forall x_{i}\in R^{c},$ by the revised condition. 

$\implies R^{\prime }\cap R^{c}=\emptyset ,R^{\prime }\subset R$

But we also have $\alpha =$ $\bigskip P_{\theta _{0}}(X\in R^{\prime })=$ $%
\bigskip P_{\theta _{0}}(X\in R)\implies \bigskip P_{\theta _{0}}(X\in
R/R^{\prime })=0$

$\bigskip P_{\theta _{1}}(X\in R^{\prime })=$ $\bigskip P_{\theta _{1}}(X\in
R)\implies \bigskip P_{\theta _{1}}(X\in R/R^{\prime })=0.$ Thus we choose $%
A=R/R^{\prime },$ and the proof is complete.

8.22$\left( b\right) \alpha =\underset{0<p\leq \frac{1}{2}}{\sup }%
P_{p}\left( \underset{i=1}{\overset{10}{\sum }}X_{i}>6\right) ,\underset{i=1}%
{\overset{10}{\sum }}X_{i}\symbol{126}B\left( 10,p\right) $

By plotting the figure of $P_{p}\left( \underset{i=1}{\overset{10}{\sum }}%
X_{i}>6\right) $ w.r.t $p$ on the interval $[0,\frac{1}{2}]$ we can

know that $\alpha =P_{\frac{1}{2}}\left( \underset{i=1}{\overset{10}{\sum }}%
X_{i}>6\right) =\frac{11}{64}.$The graph of power function is as 

follows:

\FRAME{dtbpF}{2.3125in}{1.1191in}{0pt}{}{}{Figure}{\special{language
"Scientific Word";type "GRAPHIC";display "USEDEF";valid_file "T";width
2.3125in;height 1.1191in;depth 0pt;original-width 3.7395in;original-height
2.2917in;cropleft "0";croptop "1";cropright "1";cropbottom "0";tempfilename
'NZABDH00.wmf';tempfile-properties "XPR";}}

$\left( c\right) \alpha $ can take discrete value:

0,0.0009766, 0.01074, 0.05469, 0.1719, 0.3770, 0.6230, 0.8281, 0.9453, 

0.9893, 0.9990,1.

\end{document}
