
\documentclass{article}
\usepackage{amssymb}
\usepackage{amsmath}

%%%%%%%%%%%%%%%%%%%%%%%%%%%%%%%%%%%%%%%%%%%%%%%%%%%%%%%%%%%%%%%%%%%%%%%%%%%%%%%%%%%%%%%%%%%%%%%%%%%%%%%%%%%%%%%%%%%%%%%%%%%%%%%%%%%%%%%%%%%%%%%%%%%%%%%%%%%%%%%%%%%%%%%%%%%%%%%%%%%%%%%%%%%%%%%%%%%%%%%%%%%%%%
%TCIDATA{OutputFilter=LATEX.DLL}
%TCIDATA{Version=5.00.0.2552}
%TCIDATA{<META NAME="SaveForMode" CONTENT="1">}
%TCIDATA{Created=Wednesday, September 23, 2015 21:07:41}
%TCIDATA{LastRevised=Thursday, September 24, 2015 13:47:09}
%TCIDATA{<META NAME="GraphicsSave" CONTENT="32">}
%TCIDATA{<META NAME="DocumentShell" CONTENT="Standard LaTeX\Blank - Standard LaTeX Article">}
%TCIDATA{CSTFile=40 LaTeX article.cst}

\newtheorem{theorem}{Theorem}
\newtheorem{acknowledgement}[theorem]{Acknowledgement}
\newtheorem{algorithm}[theorem]{Algorithm}
\newtheorem{axiom}[theorem]{Axiom}
\newtheorem{case}[theorem]{Case}
\newtheorem{claim}[theorem]{Claim}
\newtheorem{conclusion}[theorem]{Conclusion}
\newtheorem{condition}[theorem]{Condition}
\newtheorem{conjecture}[theorem]{Conjecture}
\newtheorem{corollary}[theorem]{Corollary}
\newtheorem{criterion}[theorem]{Criterion}
\newtheorem{definition}[theorem]{Definition}
\newtheorem{example}[theorem]{Example}
\newtheorem{exercise}[theorem]{Exercise}
\newtheorem{lemma}[theorem]{Lemma}
\newtheorem{notation}[theorem]{Notation}
\newtheorem{problem}[theorem]{Problem}
\newtheorem{proposition}[theorem]{Proposition}
\newtheorem{remark}[theorem]{Remark}
\newtheorem{solution}[theorem]{Solution}
\newtheorem{summary}[theorem]{Summary}
\newenvironment{proof}[1][Proof]{\noindent\textbf{#1.} }{\ \rule{0.5em}{0.5em}}


\begin{document}


\FRAME{ftbpF}{4.4996in}{0.7403in}{0pt}{}{}{Figure}{\special{language
"Scientific Word";type "GRAPHIC";display "USEDEF";valid_file "T";width
4.4996in;height 0.7403in;depth 0pt;original-width 6.0208in;original-height
0.896in;cropleft "0";croptop "1";cropright "1";cropbottom "0";tempfilename
'NV4TPG01.wmf';tempfile-properties "XPR";}}

Proof: 

$E\left[ g\left( X\right) \left( X-\theta \right) \right] =\sigma
^{2}E\left( g^{\prime }\left( X\right) \right) \iff \int_{-\infty }^{\infty
}g\left( x\right) \left( x-\theta \right) \frac{1}{\sqrt{2\pi }\sigma }e^{-%
\frac{\left( x-\theta \right) ^{2}}{2\sigma ^{2}}}dx=\sigma
^{2}\int_{-\infty }^{\infty }\frac{d}{dx}g\left( x\right) \frac{1}{\sqrt{%
2\pi }\sigma }e^{-\frac{\left( x-\theta \right) ^{2}}{2\sigma ^{2}}}dx.$

$\left( \text{After changing of variable }z=\frac{x-\theta }{\sigma }\right)
\iff \sigma $ $\int_{-\infty }^{\infty }g\left( \sigma z+\theta \right) z%
\frac{1}{\sqrt{2\pi }}e^{-\frac{z^{2}}{2}}dz=\sigma ^{2}\int_{-\infty
}^{\infty }\frac{1}{\sigma }\frac{d}{dz}g\left( \sigma z+\theta \right) 
\frac{1}{\sqrt{2\pi }}e^{-\frac{z^{2}}{2}}dz$

$\iff $ $\int_{-\infty }^{\infty }h\left( z\right) ze^{-\frac{z^{2}}{2}%
}dz=\int_{-\infty }^{\infty }h^{\prime }\left( z\right) e^{-\frac{z^{2}}{2}%
}dz,$ where $h\left( z\right) =g\left( \sigma z+\theta \right) $ is a
differentiable function.

$\int_{-M}^{M}h\left( z\right) ze^{-\frac{z^{2}}{2}}dz=$ $%
\int_{-M}^{M}h\left( z\right) \frac{d}{dz}\left( -e^{-\frac{z^{2}}{2}%
}\right) dz=-h\left( z\right) e^{-\frac{z^{2}}{2}}|_{-M}^{M}+$ $%
\int_{-M}^{M}h^{\prime }\left( z\right) e^{-\frac{z^{2}}{2}}dz.$

$\left\vert -h\left( z\right) e^{-\frac{z^{2}}{2}}|_{-M}^{M}\right\vert
=\left\vert h\left( M\right) -h\left( -M\right) \right\vert e^{-\frac{M^{2}}{%
2}}$

$=$ $\left\vert \int_{-M}^{M}h^{\prime }\left( z\right) e^{-\frac{M^{2}}{2}%
}dz\right\vert \leq \left\vert \int_{-N}^{N}h^{\prime }\left( z\right)
dz\right\vert e^{-\frac{M^{2}}{2}}+\left\vert \int_{N}^{M}h^{\prime }\left(
z\right) e^{-\frac{z^{2}}{2}}dz\right\vert +\left\vert
\int_{-M}^{-N}h^{\prime }\left( z\right) e^{-\frac{z^{2}}{2}}dz\right\vert ,$%
where 0$<N<M.$

$E\left\vert g^{\prime }\left( X\right) \right\vert <\infty \rightarrow
E\left\vert h^{\prime }\left( z\right) \right\vert <\infty \rightarrow
\int_{-\infty }^{\infty }h^{\prime }\left( z\right) e^{-\frac{z^{2}}{2}}dz$
converges, for any $\epsilon >0,$we can find $A\left( \epsilon \right)
>0,s.t.$

for any $N_{2}>N_{1}>A\left( \epsilon \right) ,\qquad \left\vert
\int_{N_{1}}^{N_{2}}h^{\prime }\left( z\right) e^{-\frac{z^{2}}{2}%
}dz\right\vert <\frac{\epsilon }{3}$and $\left\vert
\int_{-N_{2}}^{-N_{1}}h^{\prime }\left( z\right) e^{-\frac{z^{2}}{2}%
}dz\right\vert <\frac{\epsilon }{3}.$Take a $N\left( \epsilon \right)
>A\left( \epsilon \right) ,$

and for such fixed $N,$ we can find a lower bound $M_{0}\left( \epsilon
\right) >N\left( \epsilon \right) ,$

$s.t.\left\vert \int_{-N}^{N}h^{\prime }\left( z\right) dz\right\vert e^{-%
\frac{M^{2}}{2}}<\frac{\epsilon }{3}$holds for any $M>M_{0}\left( \epsilon
\right) .$As a result, $\left\vert -h\left( z\right) e^{-\frac{z^{2}}{2}%
}|_{-M}^{M}\right\vert <\frac{\epsilon }{3}+\frac{\epsilon }{3}+\frac{%
\epsilon }{3}=\epsilon ,$

for any $M>M_{0}\left( \epsilon \right) .$That is $\underset{M->\infty }{%
\lim }$ $\left\vert -h\left( z\right) e^{-\frac{z^{2}}{2}}|_{-M}^{M}\right%
\vert =0.$

Then we take the limit  $M->\infty $ in the equality $\int_{-M}^{M}h\left(
z\right) ze^{-\frac{z^{2}}{2}}dz=-h\left( z\right) e^{-\frac{z^{2}}{2}%
}|_{-M}^{M}+$ $\int_{-M}^{M}h^{\prime }\left( z\right) e^{-\frac{z^{2}}{2}%
}dz.$

And we get the identity  $\int_{-\infty }^{\infty }h\left( z\right) ze^{-%
\frac{z^{2}}{2}}dz=\int_{-\infty }^{\infty }h^{\prime }\left( z\right) e^{-%
\frac{z^{2}}{2}}dz.$By the equivalence at the beginning of the proof, the
proof is complete.$\boxtimes $

Below is a proof of the theorem with Fubini Theorem, which is justified by
the existence of $E\left\vert g^{\prime }\left( X\right) \right\vert \left(
<\infty \right) .$

Because of this the double integral exists. The proof comes from the
following website:

http://math.stackexchange.com/questions/280297/steins-lemma-condition\qquad
\qquad \qquad\ \ \ \ 

\FRAME{dtbpF}{4.4996in}{1.7331in}{0pt}{}{}{Figure}{\special{language
"Scientific Word";type "GRAPHIC";display "USEDEF";valid_file "T";width
4.4996in;height 1.7331in;depth 0pt;original-width 8.604in;original-height
4.8542in;cropleft "0";croptop "1";cropright "1";cropbottom "0";tempfilename
'NV4VGZ04.wmf';tempfile-properties "XPR";}}

\FRAME{dtbpF}{4.4996in}{1.817in}{0pt}{}{}{Figure}{\special{language
"Scientific Word";type "GRAPHIC";display "USEDEF";valid_file "T";width
4.4996in;height 1.817in;depth 0pt;original-width 7.9476in;original-height
4.3855in;cropleft "0";croptop "1";cropright "1";cropbottom "0";tempfilename
'NV4VFF03.wmf';tempfile-properties "XPR";}}

\FRAME{dtbpF}{4.7988in}{1.5532in}{0pt}{}{}{Figure}{\special{language
"Scientific Word";type "GRAPHIC";display "USEDEF";valid_file "T";width
4.7988in;height 1.5532in;depth 0pt;original-width 6.1142in;original-height
2.7294in;cropleft "0";croptop "1";cropright "1";cropbottom "0";tempfilename
'NV4VNL05.wmf';tempfile-properties "XPR";}}

Problem 3.39

$\left( a\right) P\left( X\leq \mu \right) =$ $\int_{-\infty }^{\mu }f\left(
x|\mu ,\sigma \right) dx=\int_{-\infty }^{\mu }\frac{1}{\sigma \pi \left(
1+\left( \frac{x-\mu }{\sigma }\right) ^{2}\right) }dx=\int_{-\infty }^{0}%
\frac{1}{\pi \left( 1+z^{2}\right) }dz=\frac{1}{\pi }\arctan \left( z\right)
|_{-\infty }^{0}=\frac{1}{2};$

$P\left( X\geq \mu \right) =1-P\left( X\leq \mu \right) =\frac{1}{2}%
.\rightarrow \mu $ is a median.

$\left( b\right) $ We firstly show the conclusion holds for $\mu =0$ and $%
\sigma =1.$If $X\symbol{126}C(1,0),$

$P\left( X>1\right) =\int_{1}^{\infty }\frac{dx}{\pi \left( 1+x^{2}\right) }=%
\frac{1}{4}.P\left( X<-1\right) =\int_{-\infty }^{-1}\frac{dx}{\pi \left(
1+x^{2}\right) }=\frac{1}{4}.$

Then we use Exercies 3.38 and the property of median in $\left( a\right) .$
We can take $z_{\alpha }=1,$then $x_{\alpha }=\sigma z_{\alpha }+\mu $

=$\sigma +\mu ,$ and $P\left( X>\sigma +\mu \right) =\frac{1}{4}.$ By
symmetric propery of the median,  $P\left( X<\mu -\sigma \right) =\frac{1}{4}%
.$

Problem 3.41 has been assigned in the last coursework. 

Problem 3.45

$\left( a\right) \qquad P\left( X\geq a\right) =E\left( \chi _{\left(
a,+\infty \right) }\left( X\right) \right) \leq \frac{E\left( e^{tX}\chi
_{\left( a,+\infty \right) }\left( X\right) \right) }{e^{at}}\leq
e^{-at}E\left( e^{tX}\right) ,$as $t>0.$

$\left( b\right) \qquad P\left( X\leq a\right) =E\left( \chi _{\left(
-\infty ,a\right) }\left( X\right) \right) \leq \frac{E\left( e^{tX}\chi
_{\left( -\infty ,a\right) }\left( X\right) \right) }{e^{at}}\leq
e^{-at}E\left( e^{tX}\right) ,$as $t<0.$

$\left( c\right) \qquad h\left( t,x\right) \geq 1$ a.s.$,$as t$\geq 0$ and $%
x\geq 0.$Otherwise($h\left( t,x\right) <1,$on a positive measurable set $S,$%
by writing

$S=\underset{i=1}{\overset{\infty }{\cup }}\{x\in S,h\left( t,x\right) <1-%
\frac{1}{i}\},S_{n}=\underset{i=1}{\overset{n}{\cup }}\{x\in S,h\left(
t,x\right) <1-\frac{1}{i}\},$then $\underset{n->\infty }{\lim }S_{n}=S.%
\underset{n->\infty }{\lim }m\left( S_{n}\right) =m\left( S\right) >0,$hence
the assumption can be refined to that $h\left( t,x\right) <1-\epsilon ,$on a
positive measurable set $S^{^{\prime }}.$ We can also find a r.v. s.t. its
pdf $f\left( x\right) $ is non-negative only on $S^{^{\prime }}$. Then $%
P\left( X\geq 0\right) -E\left( h\left( t,X\right) \right)
=\int_{S^{^{\prime }}}\left( 1-h\left( t,X\right) \right) f\left( x\right)
dx>\epsilon .$A contradiction. As a result, $h\left( t,x\right) $ must
satisfy the 

condition that $h\left( t,x\right) \geq 1$ a.s.$,$as t$\geq 0$ and $x\geq 0.$

\FRAME{fphF}{4.4996in}{0.7126in}{0pt}{}{}{Figure}{\special{language
"Scientific Word";type "GRAPHIC";display "USEDEF";valid_file "T";width
4.4996in;height 0.7126in;depth 0pt;original-width 6.1358in;original-height
0.979in;cropleft "0";croptop "1";cropright "1";cropbottom "0";tempfilename
'NV60M208.wmf';tempfile-properties "XPR";}}

\bigskip Proof: we only need to verify the above inequality for $t\geq 0.$

Let $f\left( t\right) =P\left( \left\vert Z\right\vert \geq t\right) -\sqrt{%
\frac{2}{\pi }}\frac{te^{-\frac{t^{2}}{2}}}{t^{2}+1},f^{\prime }\left(
t\right) =\frac{d}{dt}\left( \int_{t}^{\infty }\frac{1}{\sqrt{2\pi }}e^{-%
\frac{x^{2}}{2}}dx-\sqrt{\frac{2}{\pi }}\frac{te^{-\frac{t^{2}}{2}}}{t^{2}+1}%
\right) $

=$\sqrt{\frac{2}{\pi }}\frac{-2t^{2}\left( 2+t^{2}\right) }{\left(
1+t^{2}\right) ^{2}e^{t^{2}/2}}\leq 0.\rightarrow f\left( t\right)
\downarrow $ as t increases. $\rightarrow f\left( t\right) \geq \underset{%
t->\infty }{\lim }f\left( t\right) =0.\rightarrow P\left( \left\vert
Z\right\vert \geq t\right) \geq \sqrt{\frac{2}{\pi }}\frac{te^{-\frac{t^{2}}{%
2}}}{t^{2}+1}.$  

Problem 3.49

$\left( a\right) $If $X\symbol{126}\Gamma \left( \alpha ,\beta \right) ,$%
then the pdf of $X$ is $p(x)=\frac{x^{\alpha -1}}{\Gamma \left( \alpha
\right) \beta ^{\alpha }}e^{-\frac{x}{\beta }}.$

To prove E$\left[ g\left( X\right) \left( X-\alpha \beta \right) \right]
=\beta E\left( Xg^{\prime }\left( X\right) \right) $ with the assumption
that $E\left\vert Xg^{\prime }\left( X\right) \right\vert <\infty ,$ we
start from the right hand.

Since $E\left\vert Xg^{\prime }\left( X\right) \right\vert <\infty ,E\left(
Xg^{\prime }\left( X\right) \right) $ exists.$\rightarrow $ $\beta E\left(
Xg^{\prime }\left( X\right) \right) =\frac{\beta \int_{0}^{\infty }x^{\alpha
}g^{\prime }\left( x\right) e^{-x/\beta }dx}{\Gamma \left( \alpha \right)
\beta ^{\alpha }}$

=$\frac{\beta \int_{0}^{\infty }g^{\prime }\left( x\right)
dx\int_{x}^{\infty }-\frac{d}{dy}y^{\alpha }e^{-y/\beta }dy}{\Gamma \left(
\alpha \right) \beta ^{\alpha }}$=$\frac{\int_{0}^{\infty }g^{\prime }\left(
x\right) dx\int_{x}^{\infty }\left( y^{\alpha }-\alpha \beta y^{\alpha
-1}\right) e^{-y/\beta }dy}{\Gamma \left( \alpha \right) \beta ^{\alpha }},$%
using Fubini's Theorem,

=$\frac{\int_{0}^{\infty }\left( y^{\alpha }-\alpha \beta y^{\alpha
-1}\right) e^{-y/\beta }dy\int_{0}^{y}g^{\prime }\left( x\right) dx}{\Gamma
\left( \alpha \right) \beta ^{\alpha }}=\frac{\int_{0}^{\infty }\left(
y-\alpha \beta \right) y^{\alpha -1}e^{-y/\beta }\left( g\left( y\right)
-g\left( 0\right) \right) dy}{\Gamma \left( \alpha \right) \beta ^{\alpha }},
$by the property of $E\left( X\right) =\alpha \beta ,$

=$\frac{\int_{0}^{\infty }\left( y-\alpha \beta \right) y^{\alpha
-1}e^{-y/\beta }g\left( y\right) dy}{\Gamma \left( \alpha \right) \beta
^{\alpha }}=$E$\left[ g\left( X\right) \left( X-\alpha \beta \right) \right]
.$

$\left( b\right) $

If $X\symbol{126}beta\left( \alpha ,\beta \right) ,$similar to $\left(
a\right) ,$we assume then $E\left\vert \left( 1-X\right) g^{\prime }\left(
X\right) \right\vert <\infty .$

Then $E\left( \left( 1-X\right) g^{\prime }\left( X\right) \right) =\frac{%
\Gamma \left( \alpha +\beta \right) }{\Gamma \left( \alpha \right) \Gamma
\left( \beta \right) }\int_{0}^{\infty }g^{\prime }\left( x\right) x^{\alpha
-1}\left( 1-x\right) ^{\beta }dx=\frac{\Gamma \left( \alpha +\beta \right) }{%
\Gamma \left( \alpha \right) \Gamma \left( \beta \right) }\int_{0}^{\infty
}g^{\prime }\left( x\right) dx\int_{x}^{\infty }-\frac{d}{dy}y^{\alpha
-1}\left( 1-y\right) ^{\beta }dy$

=$\frac{\Gamma \left( \alpha +\beta \right) }{\Gamma \left( \alpha \right)
\Gamma \left( \beta \right) }\int_{0}^{\infty }g^{\prime }\left( x\right)
dx\int_{x}^{\infty }\left( \beta -\left( \alpha -1\right) \frac{1-y}{y}%
\right) y^{\alpha -1}\left( 1-y\right) ^{\beta -1}dy$

=$\frac{\Gamma \left( \alpha +\beta \right) }{\Gamma \left( \alpha \right)
\Gamma \left( \beta \right) }\int_{0}^{\infty }\left( \beta -\left( \alpha
-1\right) \frac{1-y}{y}\right) y^{\alpha -1}\left( 1-y\right) ^{\beta
-1}dy\int_{0}^{y}g^{\prime }\left( x\right) dx,$

$\int_{0}^{\infty }\left( \beta -\left( \alpha -1\right) \frac{1-y}{y}%
\right) y^{\alpha -1}\left( 1-y\right) ^{\beta -1}dy=\int_{0}^{\infty }\beta
y^{\alpha -1}\left( 1-y\right) ^{\beta -1}dy-\int_{0}^{\infty }\left( \alpha
-1\right) y^{\alpha -2}\left( 1-y\right) ^{\beta }dy$

=$\beta \frac{\Gamma \left( \alpha \right) \Gamma \left( \beta \right) }{%
\Gamma \left( \alpha +\beta \right) }-\left( \alpha -1\right) \frac{\Gamma
\left( \alpha -1\right) \Gamma \left( \beta +1\right) }{\Gamma \left( \alpha
+\beta \right) }=0.$

Then the equality $E\left( \left( 1-X\right) g^{\prime }\left( X\right)
\right) $ simplies to $\frac{\Gamma \left( \alpha +\beta \right) }{\Gamma
\left( \alpha \right) \Gamma \left( \beta \right) }\int_{0}^{\infty }\left(
\beta -\left( \alpha -1\right) \frac{1-y}{y}\right) y^{\alpha -1}\left(
1-y\right) ^{\beta -1}g^{\prime }\left( y\right) dy$

=$E\left[ g\left( X\right) \left( \beta -\left( \alpha -1\right) \frac{1-X}{X%
}\right) \right] .$

\end{document}
