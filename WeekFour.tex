
\documentclass{article}
\usepackage{amssymb}
\usepackage{amsmath}

%%%%%%%%%%%%%%%%%%%%%%%%%%%%%%%%%%%%%%%%%%%%%%%%%%%%%%%%%%%%%%%%%%%%%%%%%%%%%%%%%%%%%%%%%%%%%%%%%%%%%%%%%%%%%%%%%%%%%%%%%%%%%%%%%%%%%%%%%%%%%%%%%%%%%%%%%%%%%%%%%%%%%%%%%%%%%%%%%%%%%%%%%%%%%%%%%%%%%%%%%%%%%%
%TCIDATA{OutputFilter=LATEX.DLL}
%TCIDATA{Version=5.00.0.2552}
%TCIDATA{<META NAME="SaveForMode" CONTENT="1">}
%TCIDATA{Created=Friday, October 09, 2015 15:38:42}
%TCIDATA{LastRevised=Saturday, October 10, 2015 22:19:29}
%TCIDATA{<META NAME="GraphicsSave" CONTENT="32">}
%TCIDATA{<META NAME="DocumentShell" CONTENT="Scientific Notebook\Blank Document">}
%TCIDATA{CSTFile=Math with theorems suppressed.cst}
%TCIDATA{PageSetup=72,72,72,72,0}
%TCIDATA{AllPages=
%F=36,\PARA{038<p type="texpara" tag="Body Text" >\hfill \thepage}
%}


\newtheorem{theorem}{Theorem}
\newtheorem{acknowledgement}[theorem]{Acknowledgement}
\newtheorem{algorithm}[theorem]{Algorithm}
\newtheorem{axiom}[theorem]{Axiom}
\newtheorem{case}[theorem]{Case}
\newtheorem{claim}[theorem]{Claim}
\newtheorem{conclusion}[theorem]{Conclusion}
\newtheorem{condition}[theorem]{Condition}
\newtheorem{conjecture}[theorem]{Conjecture}
\newtheorem{corollary}[theorem]{Corollary}
\newtheorem{criterion}[theorem]{Criterion}
\newtheorem{definition}[theorem]{Definition}
\newtheorem{example}[theorem]{Example}
\newtheorem{exercise}[theorem]{Exercise}
\newtheorem{lemma}[theorem]{Lemma}
\newtheorem{notation}[theorem]{Notation}
\newtheorem{problem}[theorem]{Problem}
\newtheorem{proposition}[theorem]{Proposition}
\newtheorem{remark}[theorem]{Remark}
\newtheorem{solution}[theorem]{Solution}
\newtheorem{summary}[theorem]{Summary}
\newenvironment{proof}[1][Proof]{\noindent\textbf{#1.} }{\ \rule{0.5em}{0.5em}}


\begin{document}


2.5.8 5.11

\bigskip $S=\sqrt{\frac{\overset{n}{\underset{i=1}{\sum }}\left( X_{i}-%
\overline{X}\right) ^{2}}{n-1},}$by Cauchy's Inequality $E^{2}\left(
S\right) \leq E\left( S^{2}\right) E\left( 1\right) =\sigma ^{2}.$The
equality holds if $S\overset{a.s.}{=}Const.$

$\therefore E\left( S\right) \leq \sigma .$

If in addition $\sigma ^{2}>0,$then $\urcorner \left( S\overset{a.s.}{=}%
Const\right) \implies E^{2}\left( S\right) <\sigma ^{2}\implies E\left(
S\right) <\sigma $

Problem 15 

$\left( a\right) \overline{X}_{n+1}=\frac{\left( \overset{n}{\underset{i=1}{%
\sum }}X_{i}\right) +X_{n+1}}{n+1}=\frac{X_{n+1}+n\overline{X}_{n}}{n+1},$

$\left( b\right) \left( n-1\right) S_{n}^{2}=\overset{n}{\underset{i=1}{\sum 
}}\left( X_{i}-\overline{X}_{n}\right) ^{2}=\overset{n}{\underset{i=1}{\sum }%
}X_{i}^{2}-n\overline{X}_{n}^{2}$

$\left( n-2\right) S_{n-1}^{2}=\overset{n-1}{\underset{i=1}{\sum }}\left(
X_{i}-\overline{X}_{n-1}\right) ^{2}=\overset{n-1}{\underset{i=1}{\sum }}%
X_{i}^{2}-\left( n-1\right) \overline{X}_{n-1}^{2}$

$\left( n-1\right) S_{n}^{2}-\left( n-2\right) S_{n-1}^{2}=X_{n}^{2}-n%
\overline{X}_{n}^{2}+\left( n-1\right) \overline{X}_{n-1}^{2}$

=$X_{n}^{2}-\frac{1}{n}\left( \overset{n}{\underset{i=1}{\sum }}X_{i}\right)
^{2}+\left( n-1\right) \overline{X}_{n-1}^{2}$

=$X_{n}^{2}-\frac{1}{n}\left( \left( n-1\right) \overline{X}%
_{n-1}+X_{n}\right) ^{2}+\left( n-1\right) \overline{X}_{n-1}^{2}$

=$\left( 1-\frac{1}{n}\right) X_{n}^{2}-\frac{2}{n}\left( n-1\right) 
\overline{X}_{n-1}X_{n}+\left( n-1\right) \left( 1-\frac{n-1}{n}\right) 
\overline{X}_{n-1}^{2}$

=$\frac{n-1}{n}\left( X_{n}^{2}-2\overline{X}_{n-1}X_{n}+\overline{X}%
_{n-1}^{2}\right) =\frac{n-1}{n}\left( X_{n}-\overline{X}_{n-1}\right) ^{2}$

\bigskip Problem 23 

\bigskip the cmf of $Z$ is denoted as $F\left( z\right) =P\left( Z<z\right) $

$=\overset{\infty }{\underset{n=1}{\sum }}P\left( Z<z,X=n\right) =\overset{%
\infty }{\underset{n=1}{\sum }}\underset{\text{conditional probability}}{%
\underbrace{P(Z|X=n<z)}}P(X=n)$

Since $Z|X=n$ is the first-order statistic from a sample of size n,its cmf
is $1-\left( 1-F_{U}\left( z\right) \right) ^{n}$

$\implies P(Z|X=n<z)=1-\left( 1-z\right) ^{n}$

$\implies F\left( z\right) =\overset{\infty }{\underset{n=1}{\sum }}\left[
1-\left( 1-z\right) ^{n}\right] P(X=n)=1-\overset{\infty }{\underset{n=1}{%
\sum }}\left( 1-z\right) ^{n}\frac{c}{n!}=1-c\left[ \overset{\infty }{%
\underset{n=0}{\sum }}\frac{\left( 1-z\right) ^{n}}{n!}-1\right] $

$=1-c\left[ e^{1-z}-1\right] ,$where $c=\frac{1}{e-1}$ and $z\in \left(
0,1\right) .$

The pmf of $Z$ can be found by differentiating $F\left( z\right) .$

Problem 25

\bigskip From the joint pdf of $\left( X_{\left( 1\right) },X_{\left(
2\right) },...,X_{\left( n\right) }\right) ,$ $p\left(
x_{1},x_{2},...,x_{n}\right) =n!\underset{i=1}{\overset{n}{\Pi }}f\left(
x_{i}\right) =n!\underset{i=1}{\overset{n}{\Pi }}\frac{a}{\theta ^{a}}%
x_{i}^{a-1},$

for $0<x_{i}<\theta ,i=1,2,...n.$

Changing of variable $U_{i}=X_{\left( i\right) }/X_{\left( i+1\right)
},i=1,2,...n-1$ and $U_{n}=X_{n}.$

Then the joint pdf of $\left( U_{1},U_{2},...U_{n}\right) $ is $p\left(
u_{1},u_{2},...u_{n}\right) =n!\left[ \underset{i=1}{\overset{n-1}{\Pi }}%
f\left( x_{i}\left( u_{1},..u_{n}\right) \right) \right] f\left(
u_{n}\right) \left\vert \frac{\partial \left( x_{1},..x_{n}\right) }{%
\partial \left( u_{1},..u_{n}\right) }\right\vert $

$n!\left[ \underset{i=1}{\overset{n-1}{\Pi }}f\left( \underset{j=i}{\overset{%
n}{\Pi }}u_{j}\right) \right] f\left( u_{n}\right) \left( \underset{i=1}{%
\overset{n-1}{\Pi }}\underset{j=i+1}{\overset{n}{\Pi }}u_{j}\right) $

$=n!\left[ \underset{i=1}{\overset{n-1}{\Pi }}\frac{a}{\theta ^{a}}\left( 
\underset{j=i}{\overset{n}{\Pi }}u_{j}\right) ^{a-1}\right] \frac{a}{\theta
^{a}}u_{n}^{a-1}\left( \underset{i=1}{\overset{n-1}{\Pi }}\underset{j=i+1}{%
\overset{n}{\Pi }}u_{j}\right) $

$=n!\left( \frac{a}{\theta ^{a}}\right) ^{n}u_{n}^{a-1}\left( \underset{i=1}{%
\overset{n-1}{\Pi }}\underset{j=i}{\overset{n}{\Pi }}u_{j}^{a-1}\underset{i=1%
}{\overset{n-1}{\Pi }}\underset{j=i+1}{\overset{n}{\Pi }}u_{j}\right) $

=$n!\left( \frac{a}{\theta ^{a}}\right) ^{n}u_{n}^{a-1}\underset{i=1}{%
\overset{n-1}{\Pi }}\left( \underset{j=i}{\overset{n}{\Pi }}u_{j}^{a-1}%
\underset{j=i+1}{\overset{n}{\Pi }}u_{j}\right) $

=$n!\left( \frac{a}{\theta ^{a}}\right) ^{n}u_{n}^{a-1}\underset{i=1}{%
\overset{n-1}{\Pi }}\left( u_{i}^{a-1}\underset{j=i+1}{\overset{n}{\Pi }}%
u_{j}^{a-1}\underset{j=i+1}{\overset{n}{\Pi }}u_{j}\right) $

=$n!\left( \frac{a}{\theta ^{a}}\right) ^{n}u_{n}^{a-1}\underset{i=1}{%
\overset{n-1}{\Pi }}\left( u_{i}^{a-1}\underset{j=i+1}{\overset{n}{\Pi }}%
u_{j}^{a}\right) $

=$n!\left( \frac{a}{\theta ^{a}}\right) ^{n}\underset{i=1}{\overset{n}{\Pi }}%
u_{i}^{a-1}\underset{i=1}{\overset{n-1}{\Pi }}\underset{j=i+1}{\overset{n}{%
\Pi }}u_{j}^{a}\left( \text{the continued product is unchangable!}\right) $

=$n!\left( \frac{a}{\theta ^{a}}\right) ^{n}\underset{i=1}{\overset{n}{\Pi }}%
u_{i}^{a-1}\underset{i=2}{\overset{n}{\Pi }}u_{i}^{a\left( i-1\right) }$

=$n!\left( \frac{a}{\theta ^{a}}\right) ^{n}u_{1}^{a-1}\underset{j=2}{%
\overset{n}{\Pi }}u_{j}^{a-1}u_{j}^{a\left( j-1\right) }$

=$n!\left( \frac{a}{\theta ^{a}}\right) ^{n}u_{1}^{a-1}\underset{j=2}{%
\overset{n}{\Pi }}u_{j}^{aj-1}$

Since $p\left( u_{1},u_{2},...u_{n}\right) $ can be written as the product
of $u_{j},j=1,..n,$it is obvious that 

$\left( U_{1},U_{2},...U_{n}\right) $ are independent.

\bigskip For $u_{1},0<u_{n}<1$ and $\int_{0}^{1}u_{1}^{a-1}du_{n}=\frac{1}{a}%
\implies p\left( u_{1}\right) =au_{1}^{a-1};$

For $u_{n},0<u_{n}<\theta $ and $\int_{0}^{\theta }u_{n}^{an-1}du_{n}=\frac{1%
}{an}\theta ^{an}\implies p\left( u_{n}\right) =\frac{an}{\theta ^{an}}%
u_{n}^{an-1};$

For $u_{i},i=2,..n-1;0<u_{i}<1$ and $\int_{0}^{1}u_{i}^{ai-1}du_{i}=\frac{1}{%
ai}\implies p\left( u_{i}\right) =ai$ $u_{i}^{ai-1};$

It can be verified that $\underset{j=1}{\overset{n}{\Pi }}p\left(
u_{j}\right) =n!\left( \frac{a}{\theta ^{a}}\right) ^{n}u_{1}^{a-1}\underset{%
j=2}{\overset{n}{\Pi }}u_{j}^{aj-1}=$ $p\left( u_{1},u_{2},...u_{n}\right) .$

Problem 26$\left( a\right) P\left( U=u,V=v,n-U-V=w\right) =\binom{n}{u}%
F_{X}\left( u\right) ^{u}\binom{n-u}{v}\left( F_{X}\left( v\right)
-F_{X}\left( u\right) \right) ^{v}\left( 1-F_{X}\left( v\right) \right) ^{w}$

$=\frac{n!}{u!v!w!}F_{X}\left( u\right) ^{u}\left( F_{X}\left( v\right)
-F_{X}\left( u\right) \right) ^{v}\left( 1-F_{X}\left( v\right) \right) ^{w},
$where $u+v+w=n.$

$\therefore $the pmf of $\left( U,V,n-U-V\right) $ is that of a multinomial
distribution with 3 trial,

and the cell probabilities are $\left( F_{X}\left( u\right) ,F_{X}\left(
v\right) -F_{X}\left( u\right) ,1-F_{X}\left( v\right) \right) $.

$\left( b\right) $

\bigskip Notice that $F_{X\left( i\right) ,X\left( j\right) }\left(
u,v\right) =P\left( X_{\left( i\right) }\leq u,X_{\left( j\right) }\leq
v\right) $ and $U+V$ represents the number of $X_{i}$ which is smaller than $%
v.\therefore $the first equality holds. 

As for the second equality, notice that $P\left( U\geq i,U+V\geq j\right) $

=$P\left( U\geq i,U+V\geq j,U<j\right) +P\left( U\geq i,U+V\geq j,U\geq
j\right) $

$=\overset{j-1}{\underset{k=i}{\sum }}P\left( U=k,V\geq j-k\right) +P\left(
U\geq j\right) ,$since $U+V\leq n$

\bigskip =$\overset{j-1}{\underset{k=i}{\sum }}\overset{n-k}{\underset{m=j-k}%
{\sum }}P\left( U=k,V=m\right) +P\left( U\geq j\right) ,$from the
distribution solved in $\left( a\right) $

=$\frac{n!}{k!m!\left( n-k-m\right) !}F_{X}\left( u\right) ^{k}\left(
F_{X}\left( v\right) -F_{X}\left( u\right) \right) ^{m}\left( 1-F_{X}\left(
v\right) \right) ^{n-k-m}+P\left( U\geq j\right) \boxtimes $

$\left( c\right) $ $\frac{\partial ^{2}F}{\partial u\partial v}$

$=\frac{\partial }{\partial u}\underset{k=i}{\overset{j-1}{\sum }}\underset{%
m=j-k}{\overset{n-k}{\sum }}\frac{n!}{k!m!\left( n-k-m\right) !}F\left(
u\right) ^{k}\{mf\left( v\right) \left[ F\left( v\right) -F\left( u\right) %
\right] ^{m-1}\left( 1-F\left( v\right) \right) ^{n-k-m}-\left( n-k-m\right)
f\left( v\right) \left[ F\left( v\right) -F\left( u\right) \right]
^{m}\left( 1-F\left( v\right) \right) ^{n-k-m-1}\}$

$=\frac{\partial }{\partial u}\underset{k=i}{\overset{j-1}{\sum }}\frac{n!}{%
k!}F\left( u\right) ^{k}f\left( v\right) \{\underset{m=j-k}{\overset{n-k}{%
\sum }}\frac{1}{\left( m-1\right) !\left( n-k-m\right) !}\left[ F\left(
v\right) -F\left( u\right) \right] ^{m-1}\left( 1-F\left( v\right) \right)
^{n-k-m}-\underset{m=j-k}{\overset{n-k-1}{\sum }}\frac{1}{m!\left(
n-k-m-1\right) !}\left[ F\left( v\right) -F\left( u\right) \right]
^{m}\left( 1-F\left( v\right) \right) ^{n-k-m-1}\}$

$\underset{m=j-k}{\overset{n-k}{\sum }}\frac{1}{\left( m-1\right) !\left(
n-k-m\right) !}\left[ F\left( v\right) -F\left( u\right) \right]
^{m-1}\left( 1-F\left( v\right) \right) ^{n-k-m}-\underset{m=j-k}{\overset{%
n-k-1}{\sum }}\frac{1}{m!\left( n-k-m-1\right) !}\left[ F\left( v\right)
-F\left( u\right) \right] ^{m}\left( 1-F\left( v\right) \right) ^{n-k-m-1}$

=$\underset{m=j-k}{\overset{n-k}{\sum }}\frac{1}{\left( m-1\right) !\left(
n-k-m\right) !}\left[ F\left( v\right) -F\left( u\right) \right]
^{m-1}\left( 1-F\left( v\right) \right) ^{n-k-m}-\underset{m=j-k+1}{\overset{%
n-k}{\sum }}\frac{1}{\left( m-1\right) !\left( n-k-m\right) !}\left[ F\left(
v\right) -F\left( u\right) \right] ^{m-1}\left( 1-F\left( v\right) \right)
^{n-k-m}$

=$\frac{1}{\left( j-k-1\right) !\left( n-j\right) !}\left[ F\left( v\right)
-F\left( u\right) \right] ^{j-k-1}\left( 1-F\left( v\right) \right) ^{n-j}$

$\implies \frac{\partial ^{2}F}{\partial u\partial v}=\frac{\partial }{%
\partial u}\underset{k=i}{\overset{j-1}{\sum }}\frac{n!}{k!}F\left( u\right)
^{k}f\left( v\right) \frac{1}{\left( j-k-1\right) !\left( n-j\right) !}\left[
F\left( v\right) -F\left( u\right) \right] ^{j-k-1}\left( 1-F\left( v\right)
\right) ^{n-j}$

=$\frac{n!}{\left( j-1\right) !\left( n-j\right) !}f\left( v\right) \left(
1-F\left( v\right) \right) ^{n-j}\frac{\partial }{\partial u}\underset{k=i}{%
\overset{j-1}{\sum }}\binom{j-1}{k}F\left( u\right) ^{k}\left[ F\left(
v\right) -F\left( u\right) \right] ^{j-k-1}$

Similar to the derivation of pdf of $X_{\left( i\right) },$ we can show that

$\frac{\partial }{\partial u}\underset{k=i}{\overset{j-1}{\sum }}\binom{j-1}{%
k}F\left( u\right) ^{k}\left[ F\left( v\right) -F\left( u\right) \right]
^{j-k-1}=\frac{\left( j-1\right) !}{\left( i-1\right) \left( j-i-1\right) !}%
f\left( u\right) F\left( u\right) ^{i-1}\left[ F\left( v\right) -F\left(
u\right) \right] ^{j-i-1}$

$\implies \frac{\partial ^{2}F}{\partial u\partial v}=\frac{n!}{\left(
j-1\right) !\left( n-j\right) !}f\left( v\right) \left( 1-F\left( v\right)
\right) ^{n-j}\frac{\left( j-1\right) !}{\left( i-1\right) \left(
j-i-1\right) !}f\left( u\right) F\left( u\right) ^{i-1}\left[ F\left(
v\right) -F\left( u\right) \right] ^{j-i-1}$

=$\frac{n!}{\left( i-1\right) !\left( j-i-1\right) !\left( n-j\right) !}%
f\left( u\right) f\left( v\right) F\left( u\right) ^{i-1}\left[ F\left(
v\right) -F\left( u\right) \right] ^{j-i-1}\left( 1-F\left( v\right) \right)
^{n-j}\boxtimes $

\bigskip Problem 28

$\left( a\right) $ We can use the heuristic method to find the pdf of $%
X_{\left( i_{1}\right) },...,X_{\left( i_{l}\right) \ }.$

$p\left( x_{1},...,x_{l}\right) dx_{1}..x_{l}$ means the probability that
the $r$-th order statistics $X_{\left( i_{r}\right) \ }$falls within $x_{r}%
\symbol{126}x_{r}+dx_{r},$

Thus the $l$ order statistics has fixed position (see figure below)$:$

\FRAME{dtbpF}{3.5976in}{1.3033in}{0pt}{}{}{Figure}{\special{language
"Scientific Word";type "GRAPHIC";display "USEDEF";valid_file "T";width
3.5976in;height 1.3033in;depth 0pt;original-width 4.3232in;original-height
2.1145in;cropleft "0";croptop "1";cropright "1";cropbottom "0";tempfilename
'NVZSSG0C.wmf';tempfile-properties "XPR";}}

Notice that we divide the number axis by $l$ points into $l$+1 intervals, at 
$l$+1 intervals and $l$ points there is probability that one point $x_{i}$
falls into. By applying the multiplication principle we can calculate the
probabilitys that "there are $i_{1}$-1 points in $\left( -\infty
,x_{1}\right) ,$one point at $\left( x_{1},x_{1}+dx_{1}\right) ,i_{2}-i_{1}-1
$ points in $\left( x_{1},x_{2}\right) ..."P=F\left( x_{1}\right)
^{i_{1}-1}f\left( x_{1}\right) dx_{1}\left[ F\left( x_{2}\right) -F\left(
x_{1}\right) \right] ^{i_{2}-i_{1}-1}...$

$=\overset{l}{\underset{j=1}{\Pi }}f\left( x_{j}\right) \left\{ \overset{l-1}%
{\underset{j=1}{\Pi }}\left[ F\left( x_{j+1}\right) -F\left( x_{j}\right) %
\right] ^{i_{j+1}-i_{j}-1}\right\} F\left( x_{1}\right) ^{i_{1}-1}\left(
1-F\left( x_{l}\right) \right) ^{n-i_{l}}dx_{1}..x_{l}$

But we also need taking the sequence into consideration, for 2$l+1$ pieces
it requires multiplying $P$ by

$\frac{n!}{\left( i_{1}-1\right) !1!\left( i_{2}-i_{1}-1\right) !1!...\left(
n-i_{l}\right) !}$

$\implies p\left( x_{1},...,x_{l}\right) dx_{1}..x_{l}=\frac{n!}{\left(
i_{1}-1\right) !1!\left( i_{2}-i_{1}-1\right) !1!...\left( n-i_{l}\right) !}P
$

$\implies p\left( x_{1},...,x_{l}\right) =\frac{n!}{\left( i_{1}-1\right)
!\left( i_{2}-i_{1}-1\right) !...\left( n-i_{l}\right) !}\overset{l}{%
\underset{j=1}{\Pi }}f\left( x_{j}\right) \left\{ \overset{l-1}{\underset{j=1%
}{\Pi }}\left[ F\left( x_{j+1}\right) -F\left( x_{j}\right) \right]
^{i_{j+1}-i_{j}-1}\right\} F\left( x_{1}\right) ^{i_{1}-1}\left( 1-F\left(
x_{l}\right) \right) ^{n-i_{l}}$

We can use the methods in Problem 26 to calculate the cdf of $X_{\left(
i_{1}\right) },...,X_{\left( i_{l}\right) \ }.$

Let $U_{j}$ be the number of points which is less than $x_{j}.$Then $\left(
U_{1},U_{2}-U_{1},...U_{l}-U_{l-1},n-U_{l}\right) $

is a multinomial random vector with n trials and cell probability $\left(
F\left( x_{1}\right) ,F\left( x_{2}\right) -F\left( x_{1}\right) ,..F\left(
x_{l}\right) -F\left( x_{l-1}\right) ,1-F\left( x_{l}\right) \right) .$

$F\left( x_{1},...x_{l}\right) =P\left( X_{\left( i_{1}\right) }\leq
x_{1},...,X_{\left( i_{l}\right) }\leq x_{l}\right) $

$=P\left( U_{1}\geq i_{1},...U_{l}\geq i_{l}\right) =\overset{l-1}{\underset{%
k_{1}=1}{\sum }}P\left( i_{k_{1}}\leq U_{1}<i_{k_{1}+1},U_{k_{1}+1}\geq
i_{k_{1}+1}...U_{l}\geq i_{l}\right) +P\left( U_{1}\geq i_{l}\right) $

\QTP{Body Math}
$=\overset{l-1}{\underset{k_{1}=1}{\sum }}\overset{i_{k_{1}+1}-1}{\underset{%
r_{1}=i_{k_{1}}}{\sum }}P\left( U_{1}=r_{1},U_{k_{1}+1}\geq
i_{k_{1}+1}...U_{l}\geq i_{l}\right) +\overset{n}{\underset{k_{1}=i_{l}}{%
\sum }}P\left( U_{1}=k_{1}\right) $

\QTP{Body Math}
$=\overset{l-1}{\underset{k_{1}=1}{\sum }}\overset{i_{k_{1}+1}-1}{\underset{%
r_{1}=i_{k_{1}}}{\sum }}P\left( U_{k_{1}+1}-U_{1}\geq
i_{k_{1}+1}-r_{1}...U_{l}-U_{1}\geq i_{l}-r_{1}\right) +\overset{n}{\underset%
{k_{1}=i_{l}}{\sum }}\binom{n}{k_{1}}F\left( x_{1}\right) ^{k1}\left(
1-F\left( x_{1}\right) \right) ^{n-k_{1}}$

\bigskip 

For the first term, notice that we reduce the term $U_{1}\geq
i_{1},...U_{l}\geq i_{l}$ by at least one(when $k_{1}=1)$

and by induction, if we know $P\left( U_{1}\geq i_{1},...U_{l-1}\geq
i_{l-1}\right) $ in advance, then notice that 

$\left( U_{k_{1}+1}-U_{1},U_{k_{1}+2}-U_{k_{1}+1},U_{l}-U_{l-1},n-\left(
U_{l}-U_{1}\right) \right) $ is a multinomial random vector with n trials
and cell probability $\left( F\left( x_{k_{1}+1}\right) -F\left(
x_{1}\right) ,F\left( x_{k_{1}+2}\right) -F\left( x_{k_{1}+1}\right)
,..F\left( x_{l}\right) -F\left( x_{l-1}\right) ,1-\left( F\left(
x_{l}\right) -F\left( x_{1}\right) \right) \right) .$Then

we can use the induction assumption to calculate $P\left(
U_{k_{1}+1}-U_{1}\geq i_{k_{1}+1}-r_{1}...U_{l}-U_{1}\geq i_{l}-r_{1}\right)
.$

The explicit form of cmf of $X_{\left( i_{1}\right) },...,X_{\left(
i_{l}\right) \ }$is very hard to get.

$\left( b\right) $We can calculate the conditional pdf of $X_{\left(
i_{1}\right) },...,X_{\left( i_{l}\right) \ }$by formula

$p\left( x_{1},...,x_{l}|x_{1}^{\prime },...,x_{m}^{\prime }\right) =\frac{%
p\left( x_{1},...,x_{l},x_{1}^{\prime },...,x_{m}^{\prime }\right) }{p\left(
x_{1}^{\prime },...,x_{m}^{\prime }\right) },$where $p\left( x_{1}^{\prime
},...,x_{l}^{\prime }\right) $ is the pdf of $X_{\left( j_{1}\right)
},...,X_{\left( j_{m}\right) \ }$and

$p\left( x_{1},...,x_{l},x_{1}^{\prime },...,x_{m}^{\prime }\right) $ is the
joint pdf of $X_{\left( i_{1}\right) },...,X_{\left( i_{l}\right) \
},X_{\left( j_{1}\right) },...,X_{\left( j_{m}\right) \ }.$

For the conditional cmf we must presume that $X_{\left( j_{1}\right)
},...,X_{\left( j_{m}\right) \ }$each lies within a small interval and not
at a single point. $F\left( x_{1},...,x_{l}|x_{1}^{\prime }\leq X_{\left(
j_{1}\right) }\leq x_{1}^{\prime }+\Delta x_{1}^{\prime },...,x_{m}^{\prime
}\leq X_{\left( j_{m}\right) }\leq x_{m}^{\prime }+\Delta x_{m}^{\prime
}\right) $

$P\left( X_{\left( i_{1}\right) }<x_{1},...,X_{\left( i_{l}\right) \
}<x_{l}|x_{1}^{\prime }\leq X_{\left( j_{1}\right) }\leq x_{1}^{\prime
}+\Delta x_{1}^{\prime },...,x_{m}^{\prime }\leq X_{\left( j_{m}\right)
}\leq x_{m}^{\prime }+\Delta x_{m}^{\prime }\right) $

$=\frac{P\left( X_{\left( i_{1}\right) }<x_{1},...,X_{\left( i_{l}\right) \
}<x_{l},x_{1}^{\prime }\leq X_{\left( j_{1}\right) }\leq x_{1}^{\prime
}+\Delta x_{1}^{\prime },...,x_{m}^{\prime }\leq X_{\left( j_{m}\right)
}\leq x_{m}^{\prime }+\Delta x_{m}^{\prime }\right) }{P\left( x_{1}^{\prime
}\leq X_{\left( j_{1}\right) }\leq x_{1}^{\prime }+\Delta x_{1}^{\prime
},...,x_{m}^{\prime }\leq X_{\left( j_{m}\right) }\leq x_{m}^{\prime
}+\Delta x_{m}^{\prime }\right) },$then we can use the known cdf to
calculate the probability.

However, such work is daunting in practice and the exact formula for
conditional cmf is also hard to get.

\bigskip 

\end{document}
