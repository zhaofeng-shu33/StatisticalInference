
\documentclass{article}
\usepackage{amsmath}

%%%%%%%%%%%%%%%%%%%%%%%%%%%%%%%%%%%%%%%%%%%%%%%%%%%%%%%%%%%%%%%%%%%%%%%%%%%%%%%%%%%%%%%%%%%%%%%%%%%%%%%%%%%%%%%%%%%%%%%%%%%%%%%%%%%%%%%%%%%%%%%%%%%%%%%%%%%%%%%%%%%%%%%%%%%%%%%%%%%%%%%%%%%%%%%%%%%%%%%%%%%%%%%%%%%%%%%%%%%%%%%%%%%%
%TCIDATA{OutputFilter=LATEX.DLL}
%TCIDATA{Version=5.00.0.2552}
%TCIDATA{<META NAME="SaveForMode" CONTENT="1">}
%TCIDATA{Created=Sunday, December 06, 2015 22:40:32}
%TCIDATA{LastRevised=Sunday, December 06, 2015 23:17:02}
%TCIDATA{<META NAME="GraphicsSave" CONTENT="32">}
%TCIDATA{<META NAME="DocumentShell" CONTENT="Scientific Notebook\Blank Document">}
%TCIDATA{CSTFile=Math with theorems suppressed.cst}
%TCIDATA{PageSetup=72,72,72,72,0}
%TCIDATA{AllPages=
%F=36,\PARA{038<p type="texpara" tag="Body Text" >\hfill \thepage}
%}


\newtheorem{theorem}{Theorem}
\newtheorem{acknowledgement}[theorem]{Acknowledgement}
\newtheorem{algorithm}[theorem]{Algorithm}
\newtheorem{axiom}[theorem]{Axiom}
\newtheorem{case}[theorem]{Case}
\newtheorem{claim}[theorem]{Claim}
\newtheorem{conclusion}[theorem]{Conclusion}
\newtheorem{condition}[theorem]{Condition}
\newtheorem{conjecture}[theorem]{Conjecture}
\newtheorem{corollary}[theorem]{Corollary}
\newtheorem{criterion}[theorem]{Criterion}
\newtheorem{definition}[theorem]{Definition}
\newtheorem{example}[theorem]{Example}
\newtheorem{exercise}[theorem]{Exercise}
\newtheorem{lemma}[theorem]{Lemma}
\newtheorem{notation}[theorem]{Notation}
\newtheorem{problem}[theorem]{Problem}
\newtheorem{proposition}[theorem]{Proposition}
\newtheorem{remark}[theorem]{Remark}
\newtheorem{solution}[theorem]{Solution}
\newtheorem{summary}[theorem]{Summary}
\newenvironment{proof}[1][Proof]{\noindent\textbf{#1.} }{\ \rule{0.5em}{0.5em}}
\input{tcilatex}

\begin{document}


8.14 Take $R=\left\{ \bar{x}|\bar{x}>0.5\right\} .$

As n is very large $\frac{\Sigma X_{i}-np}{\sqrt{np\left( 1-p\right) }}%
\symbol{126}N\left( 0,1\right) $ approximately.

$\implies \frac{\overline{X}-p}{\sqrt{p\left( 1-p\right) /n}}\symbol{126}%
N\left( 0,1\right) .$ The probability of making the first kind of error:

$P_{p=0.49}\left( \overline{X}\in R\right) =P_{p=0.49}\left( \overline{X}%
>0.5\right) =P\left( Z>0.01\sqrt{\frac{n}{0.49\times 0.51}}\right) ,$where $Z%
\symbol{126}N\left( 0,1\right) .$

Also the probability of making the second kind of error:

$P_{p=0.51}\left( \overline{X}\notin R\right) =P_{p=0.51}\left( \overline{X}%
\leq 0.5\right) =P\left( Z\leq -0.01\sqrt{\frac{n}{0.49\times 0.51}}\right) .
$

By the symmetry of cdf of $Z\implies 0.01\sqrt{\frac{n}{0.49\times 0.51}}=$%
quantile$\left( 0.99\right) $

Numerical approximation gives:$n\approx 13524.$

8.16 $\left( 1\right) $ In this case the rejection region is the whole space.

$\alpha =\underset{\theta \in \Theta }{\sup }\beta \left( \theta \right) =%
\underset{\theta \in \Theta }{\sup }P_{\theta }\left( X\in R\right) =1.$ $%
\beta \left( \theta \right) \equiv 1.$ Though the probability of making the
second kind of error is zero, the probability of making the first kind of
error is uncontrolled.

$\left( 2\right) $ $\alpha =0,\beta \left( \theta \right) \equiv 0.$ Though
the probability of making the first kind of error is zero, the probability
of making the second kind of error is uncontrolled.

8.18 $\left( 1\right) \beta \left( \theta \right) =P_{\theta }\left( \frac{%
\overline{X}-\theta _{0}}{\sigma /\sqrt{n}}>c\right) =P_{\theta }\left( 
\frac{\overline{X}-\theta _{0}}{\sigma /\sqrt{n}}>c\right) +P_{\theta
}\left( \frac{\overline{X}-\theta _{0}}{\sigma /\sqrt{n}}<-c\right) $

$=P\left( \frac{\overline{X}-\theta }{\sigma /\sqrt{n}}>c+\frac{\theta
_{0}-\theta }{\sigma /\sqrt{n}}\right) +P\left( \frac{\overline{X}-\theta }{%
\sigma /\sqrt{n}}<-c+\frac{\theta _{0}-\theta }{\sigma /\sqrt{n}}\right) $

$=P\left( Z>c+\frac{\theta _{0}-\theta }{\sigma /\sqrt{n}}\right) +P\left(
Z<-c+\frac{\theta _{0}-\theta }{\sigma /\sqrt{n}}\right) ,$where $Z\symbol{%
126}N\left( 0,1\right) $

$=1-\Phi \left( c+\frac{\theta _{0}-\theta }{\sigma /\sqrt{n}}\right) +\Phi
\left( -c+\frac{\theta _{0}-\theta }{\sigma /\sqrt{n}}\right) ,$where $\Phi $
is the cdf of standard normal distribution.

$\left( 2\right) \beta \left( \theta _{0}\right) =0.05,1-\beta \left( \theta
_{0}+\sigma \right) \leq 0.25.$

$1-\Phi \left( c\right) +\Phi \left( -c\right) =0.05\implies 2\Phi \left(
-c\right) =0.05\implies c=1.96.$

$1-\beta \left( \theta _{0}+\sigma \right) =\Phi \left( c+\frac{-\sigma }{%
\sigma /\sqrt{n}}\right) -\Phi \left( -c-\frac{\sigma }{\sigma /\sqrt{n}}%
\right) $

$=\Phi \left( c-\sqrt{n}\right) -\Phi \left( -c-\sqrt{n}\right) \leq 0.25$

Numerical solution gives: $n\geq 8.$

\end{document}
