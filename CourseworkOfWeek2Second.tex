
\documentclass{article}
%%%%%%%%%%%%%%%%%%%%%%%%%%%%%%%%%%%%%%%%%%%%%%%%%%%%%%%%%%%%%%%%%%%%%%%%%%%%%%%%%%%%%%%%%%%%%%%%%%%%%%%%%%%%%%%%%%%%%%%%%%%%%%%%%%%%%%%%%%%%%%%%%%%%%%%%%%%%%%%%%%%%%%%%%%%%%%%%%%%%%%%%%%%%%%%%%%%%%%%%%%%%%%%%%%%%%%%%%%%%%%%%%%%%%%%%%%%%%%%%%%%%%%%%%%%%
\usepackage{amsmath}

\setcounter{MaxMatrixCols}{10}
%TCIDATA{OutputFilter=LATEX.DLL}
%TCIDATA{Version=5.00.0.2552}
%TCIDATA{<META NAME="SaveForMode" CONTENT="1">}
%TCIDATA{Created=Friday, September 25, 2015 17:18:40}
%TCIDATA{LastRevised=Saturday, October 10, 2015 22:22:28}
%TCIDATA{<META NAME="GraphicsSave" CONTENT="32">}
%TCIDATA{<META NAME="DocumentShell" CONTENT="Standard LaTeX\Blank - Standard LaTeX Article">}
%TCIDATA{CSTFile=40 LaTeX article.cst}
%TCIDATA{ComputeDefs=
%$X$
%}


\newtheorem{theorem}{Theorem}
\newtheorem{acknowledgement}[theorem]{Acknowledgement}
\newtheorem{algorithm}[theorem]{Algorithm}
\newtheorem{axiom}[theorem]{Axiom}
\newtheorem{case}[theorem]{Case}
\newtheorem{claim}[theorem]{Claim}
\newtheorem{conclusion}[theorem]{Conclusion}
\newtheorem{condition}[theorem]{Condition}
\newtheorem{conjecture}[theorem]{Conjecture}
\newtheorem{corollary}[theorem]{Corollary}
\newtheorem{criterion}[theorem]{Criterion}
\newtheorem{definition}[theorem]{Definition}
\newtheorem{example}[theorem]{Example}
\newtheorem{exercise}[theorem]{Exercise}
\newtheorem{lemma}[theorem]{Lemma}
\newtheorem{notation}[theorem]{Notation}
\newtheorem{problem}[theorem]{Problem}
\newtheorem{proposition}[theorem]{Proposition}
\newtheorem{remark}[theorem]{Remark}
\newtheorem{solution}[theorem]{Solution}
\newtheorem{summary}[theorem]{Summary}
\newenvironment{proof}[1][Proof]{\noindent\textbf{#1.} }{\ \rule{0.5em}{0.5em}}


\begin{document}


Problem5.2\qquad $\left( 1\right) $Let r.v.$Y$ represent the number of years
passed until the first year's rainfall is exceeded.

Then $P\left( Y=1\right) =0;P\left( Y=2\right) =P\left( X_{2}>X_{1}\right) ;$%
since $X_{2}$ and $X_{1}$ have the same pdf. By symmetry, \ $\qquad \
P\left( X_{2}>X_{1}\right) =P\left( X_{2}\leq X_{1}\right) =\frac{1}{2}%
.P\left( Y=3\right) =P\left( X_{2}\leq X_{1}\cap X_{3}>X_{1}\right) $

$=P\left( X_{2}\leq X_{1}\leq X_{3}\right) \qquad $

By symmetry of $X_{1},X_{2},X_{3},$which has 6 kinds of possibility of
permutation.

$\rightarrow P\left( X_{2}\leq X_{1}\cup X_{3}>X_{1}\right) =\frac{1}{6}%
\qquad $that is $P\left( Y=3\right) =\frac{1}{6}.$

\bigskip $P\left( Y=4\right) =P\left( \left( X_{2}\leq X_{1}\right) \cap
\left( X_{3}\leq X_{1}\right) \cap \left( X_{4}>X_{1}\right) \right) ,$to
calculate this value, we can permutate

$X_{1},X_{2},X_{3}$,$X_{4}$ in such a way that in one ordered sequence, the
last two elements are orderless,eg.

$X_{1},X_{2},X_{3}$,$X_{4}$ and $X_{1},X_{2},X_{4}$,$X_{3}$ are one kind of
permutation. Then we have $\frac{4!}{2}=12$ kinds in total.

And it follows $P\left( Y=4\right) =\frac{1}{12}.$

By induction, $P\left( Y=n\right) =\frac{1}{\frac{n!}{\left( n-2\right) !}}=%
\frac{1}{n\left( n-1\right) }$for $n\geq 2.$The pmf of $Y$ is obtained$.$

Remark: For $Y=n,$ we calculate the probability that $X_{\max \left\{
1,2...n-1\right\} }\leq X_{1}$ {<}$X_{n}$and 

$\left( 2\right) $ $E\left( Y\right) =\underset{n=2}{\overset{\infty }{\sum }%
}n\frac{1}{n\left( n-1\right) }=\infty .$

Problem5.3

$Y_{i}\symbol{126}Bernoulli\left( 1-F_{X}\left( \mu \right) \right)
\rightarrow \underset{i=1}{\overset{n}{\sum }}Y_{i}\symbol{126}B\left(
n,1-F_{X}\left( \mu \right) \right) .$

\bigskip Problem5.4

$\left( a\right) P\left( X_{1}=x_{1},...,X_{k}=x_{k}\right) =P\left(
X_{1}=x_{1},...,X_{k}=x_{k},0<P<1\right) $

$=\int_{0}^{1}P\left( \left( X_{1}=x_{1},...,X_{k}=x_{k}\right) |P=x\right)
dx=\int_{0}^{1}P\left( \left( X_{1}=x_{1}|P=x,...,X_{k}=x_{k}|P=x\right)
\right) dx$

=$\int_{0}^{1}P\left( \left( X_{1}=x_{1}|P=x,...,X_{k}=x_{k}|P=x\right)
\right) dx=\int_{0}^{1}P\left( X_{1}=x_{1}|P=x)...P(X_{k}=x_{k}|P=x\right)
dx $

$=\int_{0}^{1}\underset{i=1}{\overset{k}{\Pi }}x^{x_{i}}\left( 1-x\right)
^{1-x_{i}}dx=\int_{0}^{1}x^{t}\left( 1-x\right) ^{k-t}dx,$where $t=\underset{%
i=1}{\overset{k}{\sum }}x_{i}.$

$\left( b\right) P\left( X_{i}=x_{i}\right)
=\int_{0}^{1}P(X_{i}=x_{i}|P=x)dx=\int_{0}^{1}x^{x_{i}}\left( 1-x\right)
^{1-x_{i}}dx.$

$\int_{0}^{1}x^{t}\left( 1-x\right) ^{k-t}dx\neq \underset{i=1}{\overset{k}{%
\Pi }}\int_{0}^{1}x^{x_{i}}\left( 1-x\right) ^{1-x_{i}}dx$ can be seen if $%
x_{i}=1,i=1,2...k.$

that is $\frac{1}{k+1}\neq \frac{1}{2^{k}}.$

\bigskip

Added Coursework:

Problem 1:

\bigskip Let $X_{n}=O_{p}\left( 1\right) ,Y_{n}=o_{p}\left( 1\right) .$

$\left( a\right) $From $Y_{n}=O_{p}\left( 1\right) $ follows $\forall \delta
>0,\exists N_{\delta },M_{\delta }$ $s.t.P\left( \left\vert Y_{n}\right\vert
>M_{\delta }\right) <\frac{\delta }{2},$for $n\geq N_{\delta }$

$\forall \epsilon >0\qquad P\left( \left\vert X_{n}Y_{n}\right\vert
>\epsilon \right) =P\left( \left\vert X_{n}Y_{n}\right\vert >\epsilon
,\left\vert Y_{n}\right\vert >M_{\delta }\right) +P\left( \left\vert
X_{n}Y_{n}\right\vert >\epsilon ,\left\vert Y_{n}\right\vert \leq M_{\delta
}\right) $

$\leq \frac{\delta }{2}+P\left( \left\vert X_{n}\right\vert >\frac{\epsilon 
}{M_{\delta }}\right) $

\bigskip Since $X_{n}\overset{p}{\rightarrow }0,\exists N_{\epsilon ,\delta
}>N_{\delta }$ $s.t.P\left( \left\vert X_{n}\right\vert >\frac{\epsilon }{%
M_{\delta }}\right) <\frac{\delta }{2},$for $n\geq N_{\epsilon ,\delta }.$

Then$\forall \epsilon >0(\forall \delta >0$,$\exists N_{\epsilon
}\rightarrow \forall n\geq N_{\epsilon ,\delta }(P\left( \left\vert
X_{n}Y_{n}\right\vert >\epsilon \right) <\frac{\delta }{2}+\frac{\delta }{2}%
=\allowbreak \delta )).$

\bigskip $X_{n}Y_{n}\overset{p}{\rightarrow }0$ follows, which is equivalent
to $O_{p}\left( 1\right) o_{p}\left( 1\right) =o_{p}\left( 1\right) .$

$\left( b\right) \forall \epsilon >0,$from $Y_{n}=O_{p}\left( 1\right) $
follows$:\exists N_{\epsilon },M_{\epsilon }$ $s.t.P\left( \left\vert
Y_{n}\right\vert >M_{\epsilon }\right) <\frac{\epsilon }{2},$for $n\geq
N_{\epsilon };$

\qquad From $X_{n}=o_{p}\left( 1\right) $ follows $\exists N_{\epsilon
}^{\prime }>N_{\epsilon }$ $s.t.P\left( \left\vert X_{n}\right\vert
>1\right) <\frac{\epsilon }{2},$for $n\geq N_{\epsilon }^{\prime };$

$P\left( \left\vert X_{n}+Y_{n}\right\vert >M_{\epsilon }+1\right) =P\left(
\left\vert X_{n}+Y_{n}\right\vert >M_{\epsilon }+1,\left\vert
X_{n}\right\vert >1\right) +P\left( \left\vert X_{n}+Y_{n}\right\vert
>M_{\epsilon }+1,\left\vert X_{n}\right\vert \leq 1\right) $

$\leq \frac{\epsilon }{2}+P\left( \left\vert Y_{n}\right\vert >M_{\epsilon
}\right) <\epsilon ,$for $n>N_{\epsilon }^{\prime }.$

$\rightarrow O_{p}\left( 1\right) +o_{p}\left( 1\right) =O_{p}\left(
1\right) .$

$\left( c\right) $

\bigskip Let the distribution function sequence of $\left\{ X_{n}\right\} $
be $\left\{ F_{n}\right\} .$

$X_{n}\overset{D}{\rightarrow }X\iff F_{n}\overset{w}{\rightarrow }F,$where $%
F$ is the distribution function of $X.$

$\iff \forall x\in C\left( F\right) ,F_{n}\left( x\right) \rightarrow
F\left( x\right) .$

$\forall \epsilon >0,\exists M_{\epsilon }\in C\left( F\right) $ and $%
-M_{\epsilon }\in C\left( F\right) $ $,s.t.F\left( M_{\epsilon }\right)
-F\left( -M_{\epsilon }\right) >1-\epsilon .$

By continuity of limit process $\exists N_{\epsilon },s.t.F_{n}\left(
M_{\epsilon }\right) -F_{n}\left( -M_{\epsilon }\right) >1-\epsilon ,\forall
n>N_{\epsilon }.$

$X_{n}=O_{p}\left( 1\right) $ follows.

$\left( d\right) $

\bigskip $\overline{X}_{n}:=\frac{1}{n}\underset{i=1}{\overset{n}{\sum }}%
X_{i}.$By Large Number Law, $\overline{X}_{n}\overset{p}{\rightarrow }\mu
\iff \overline{X}_{n}=\mu +o_{p}\left( 1\right) .$

$P\left( \left\vert \frac{1}{n}\underset{i=1}{\overset{n}{\sum }}X_{i}-\mu
\right\vert >C\right) =P\left( \left\vert \underset{i=1}{\overset{n}{\sum }}%
\left( X_{i}-\mu \right) \right\vert >nC\right) \leq \frac{D\left( \underset{%
i=1}{\overset{n}{\sum }}X_{i}\right) }{n^{2}C^{2}}=\frac{EX_{1}^{2}+E^{2}%
\left\vert X_{1}\right\vert }{nC^{2}}$

Substituting $n^{-1/2}C$ for C:

$P\left( \sqrt{n}\left\vert \frac{1}{n}\underset{i=1}{\overset{n}{\sum }}%
X_{i}-\mu \right\vert >C\right) \leq \frac{EX_{1}^{2}+E^{2}\left\vert
X_{1}\right\vert }{C^{2}}.$

$\forall \epsilon >0,$let $C=\sqrt{\frac{EX_{1}^{2}+E^{2}\left\vert
X_{1}\right\vert }{\epsilon }},$then $P\left( \sqrt{n}\left\vert \frac{1}{n}%
\underset{i=1}{\overset{n}{\sum }}X_{i}-\mu \right\vert >C\right) \leq
\epsilon .$

$\frac{1}{n}\underset{i=1}{\overset{n}{\sum }}X_{i}=\mu +O_{p}\left(
n^{-1/2}\right) $ follows.

\bigskip Problem 2

$\left( a\right) \overline{X}_{n}:=\frac{1}{n}\underset{i=1}{\overset{n}{%
\sum }}X_{i}.$

Using the equivalent condition for a.s. convergence: $\qquad \qquad \forall
\epsilon >0,\underset{k->\infty }{\lim }P\left( \underset{n=k}{\overset{%
\infty }{\cup }}\left[ \left\vert \overline{X_{n}}-p\right\vert >\epsilon %
\right] \right) =0.$

By Hoeffding's inequality, $P\left[ \left\vert \overline{X_{n}}-p\right\vert
>\epsilon \right] =P\left[ \left\vert \underset{i=1}{\overset{n}{\sum }}%
\left( X_{i}-p\right) \right\vert >n\epsilon \right] \leq 2e^{\frac{%
-2n^{2}\epsilon ^{2}}{n}}.$

By Weistrass' Test,$\underset{n=1}{\overset{\infty }{\sum }}P\left[
\left\vert \overline{X_{n}}-p\right\vert >\epsilon \right] <\infty
\rightarrow $

$\underset{k->\infty }{\lim }P\left( \underset{n=k}{\overset{\infty }{\cup }}%
\left[ \left\vert \overline{X_{n}}-p\right\vert >\epsilon \right] \right)
\leq \underset{k->\infty }{\lim }\underset{n=k}{\overset{\infty }{\sum }}%
P\left( \left[ \left\vert \overline{X_{n}}-p\right\vert >\epsilon \right]
\right) =0.$

The result follows.

\bigskip $\left( b\right) $ By defintion: $\overline{X_{n}}-p=o_{a.s.}\left( 
\sqrt{\frac{\ln n}{n}}\right) \iff \sqrt{\frac{n}{\ln n}}\left( \overline{%
X_{n}}-p\right) \overset{P}{\rightarrow }0$

From $\left( a\right) $ we have $P\left[ \left\vert \overline{X_{n}}%
-p\right\vert >\epsilon \right] \leq 2e^{-2n\epsilon ^{2}}.$

Substituting $\sqrt{\frac{\ln n}{n}}\epsilon $ for $\epsilon $ gives

$P\left[ \sqrt{\frac{n}{\ln n}}\left\vert \overline{X_{n}}-p\right\vert
>\epsilon \right] \leq \frac{2}{n^{2\epsilon ^{2}}}\rightarrow 0,$as $%
n\rightarrow \infty .$

As a result,

$\sqrt{\frac{n}{\ln n}}\left( \overline{X_{n}}-p\right) \overset{P}{%
\rightarrow }0.$

\bigskip

$\bigskip $

$\bigskip $

\end{document}
