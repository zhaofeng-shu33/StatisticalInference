
\documentclass{article}
\usepackage{amssymb}
\usepackage{amsmath}

%%%%%%%%%%%%%%%%%%%%%%%%%%%%%%%%%%%%%%%%%%%%%%%%%%%%%%%%%%%%%%%%%%%%%%%%%%%%%%%%%%%%%%%%%%%%%%%%%%%%%%%%%%%%%%%%%%%%%%%%%%%%%%%%%%%%%%%%%%%%%%%%%%%%%%%%%%%%%%%%%%%%%%%%%%%%%%%%%%%%%%%%%%%%%%%%%%%%%%%%%%%%%%
%TCIDATA{OutputFilter=LATEX.DLL}
%TCIDATA{Version=5.00.0.2552}
%TCIDATA{<META NAME="SaveForMode" CONTENT="1">}
%TCIDATA{Created=Friday, October 16, 2015 16:04:52}
%TCIDATA{LastRevised=Friday, October 16, 2015 21:03:48}
%TCIDATA{<META NAME="GraphicsSave" CONTENT="32">}
%TCIDATA{<META NAME="DocumentShell" CONTENT="Scientific Notebook\Blank Document">}
%TCIDATA{CSTFile=Math with theorems suppressed.cst}
%TCIDATA{PageSetup=72,72,72,72,0}
%TCIDATA{AllPages=
%F=36,\PARA{038<p type="texpara" tag="Body Text" >\hfill \thepage}
%}


\newtheorem{theorem}{Theorem}
\newtheorem{acknowledgement}[theorem]{Acknowledgement}
\newtheorem{algorithm}[theorem]{Algorithm}
\newtheorem{axiom}[theorem]{Axiom}
\newtheorem{case}[theorem]{Case}
\newtheorem{claim}[theorem]{Claim}
\newtheorem{conclusion}[theorem]{Conclusion}
\newtheorem{condition}[theorem]{Condition}
\newtheorem{conjecture}[theorem]{Conjecture}
\newtheorem{corollary}[theorem]{Corollary}
\newtheorem{criterion}[theorem]{Criterion}
\newtheorem{definition}[theorem]{Definition}
\newtheorem{example}[theorem]{Example}
\newtheorem{exercise}[theorem]{Exercise}
\newtheorem{lemma}[theorem]{Lemma}
\newtheorem{notation}[theorem]{Notation}
\newtheorem{problem}[theorem]{Problem}
\newtheorem{proposition}[theorem]{Proposition}
\newtheorem{remark}[theorem]{Remark}
\newtheorem{solution}[theorem]{Solution}
\newtheorem{summary}[theorem]{Summary}
\newenvironment{proof}[1][Proof]{\noindent\textbf{#1.} }{\ \rule{0.5em}{0.5em}}


\begin{document}


Coursework Of Week Five Second\bigskip 

Problem 5.40

$\left( a\right) \qquad P\left( X\leq t-\epsilon \right) =\qquad P\left(
X\leq t-\epsilon ,X_{n}\leq t\right) +\qquad P\left( X\leq t-\epsilon
,X_{n}>t\right) $

$\leq P\left( X_{n}\leq t\right) +\qquad P\left( \left\vert
X_{n}-X\right\vert \geq \epsilon \right) $

We get a lower bound on $P\left( X_{n}\leq t\right) :$

$\qquad P\left( X\leq t-\epsilon \right) -P\left( \left\vert
X_{n}-X\right\vert \geq \epsilon \right) \leq P\left( X_{n}\leq t\right) $

$\left( b\right) P\left( X_{n}\leq t\right) =\qquad P\left( X_{n}\leq
t,X\leq t+\epsilon \right) +\qquad P\left( X_{n}\leq t,X>t+\epsilon \right) $

$\leq P\left( X\leq t+\epsilon \right) +P\left( \left\vert
X_{n}-X\right\vert \geq \epsilon \right) $

We get an upper bound on $P\left( X_{n}\leq t\right) .$

$\left( c\right) $Combing $\left( a,b\right) $ we have $\forall \epsilon >0$

$P\left( X\leq t-\epsilon \right) -P\left( \left\vert X_{n}-X\right\vert
\geq \epsilon \right) \leq P\left( X_{n}\leq t\right) \leq P\left( X\leq
t+\epsilon \right) +P\left( \left\vert X_{n}-X\right\vert \geq \epsilon
\right) $

Since $X_{n}\overset{p}{\rightarrow }X,$taking the limit $n->\infty $ gives

$P\left( X\leq t-\epsilon \right) \leq \underset{n->\infty }{\lim \inf }%
P\left( X_{n}\leq t\right) \leq \underset{n->\infty }{\lim \sup }P\left(
X_{n}\leq t\right) \leq P\left( X\leq t+\epsilon \right) $

If $t\in C\left( F\left( X\right) \right) ,$then letting $\epsilon ->0$ givs

$P\left( X\leq t\right) \leq \underset{n->\infty }{\lim \inf }P\left(
X_{n}\leq t\right) \leq \underset{n->\infty }{\lim \sup }P\left( X_{n}\leq
t\right) \leq P\left( X\leq t\right) $

$\implies \underset{n->\infty }{\lim }P\left( X_{n}\leq t\right) =P\left(
X\leq t\right) .$

Problem 5.41

\bigskip Since the cdf of constant as r.v. is discontinuous at $x=\mu ,$ we
cannot claim the convergence of 

$\underset{n->\infty }{\lim }P\left( X_{n}\leq x\right) .$

$\left( a\right) P\left( \left\vert X_{n}-\mu \right\vert \leq \epsilon
\right) \leq P\left( X_{n}\leq \mu +\epsilon \right) \overset{\epsilon
=x-\mu >0}{=}P\left( X_{n}\leq \mu +x-\mu \right) =P\left( X_{n}\leq
x\right) $

Then $1-P\left( \left\vert X_{n}-\mu \right\vert >\epsilon \right) \leq
P\left( X_{n}\leq x\right) ,$let $n->\infty \implies 1\leq \underset{%
n->\infty }{\lim \inf }P\left( X_{n}\leq x\right) $

$\implies \underset{n->\infty }{\lim }P\left( X_{n}\leq x\right) =1$ if $%
x>\mu .$

$P\left( \left\vert X_{n}-\mu \right\vert \geq \epsilon \right) \geq P\left(
X_{n}\leq \mu -\epsilon \right) \overset{\epsilon =\mu -x>0}{=}P\left(
X_{n}\leq x\right) $

\bigskip $\underset{n->\infty }{\lim \sup }P\left( X_{n}\leq x\right) \leq
0\implies \underset{n->\infty }{\lim }P\left( X_{n}\leq x\right) =0.$

The $\rightarrow $ implication is deduced.

$\left( b\right) P\left( \left\vert X_{n}-\mu \right\vert >\epsilon \right)
=P\left( X_{n}<\mu -\epsilon \right) +P\left( X_{n}>\mu +\epsilon \right) =$

$P\left( X_{n}<\mu -\epsilon \right) +1-P\left( X_{n}\leq \mu +\epsilon
\right) ->0+1-1=0\boxtimes $

\bigskip 

Problem 5.43 Since \FRAME{dtbpF}{2.1456in}{0.3304in}{0pt}{}{}{Figure}{%
\special{language "Scientific Word";type "GRAPHIC";display
"USEDEF";valid_file "T";width 2.1456in;height 0.3304in;depth
0pt;original-width 3.4869in;original-height 0.4445in;cropleft "0";croptop
"1";cropright "1";cropbottom "0";tempfilename
'NWB79C0R.wmf';tempfile-properties "XPR";}}

$\left( a\right) P\left( \left\vert Y_{n}-\mu \right\vert >\epsilon \right)
=P\left( \left\vert \sqrt{n}\left( Y_{n}-\mu \right) \right\vert >\sqrt{n}%
\epsilon \right) ->0$ follows from the  uniform bounded property of $\sqrt{n}%
\left( Y_{n}-\mu \right) .$

$\left( b\right) $First we show that

if $R\left( t\right) =o\left( t-\mu \right) ,t->\mu ,.$Then $\sqrt{n}R\left(
Y_{n}\right) \overset{P}{\rightarrow }0,n->\infty .$

We can show that $R\left( Y_{n}\right) =o_{p}\left( Y_{n}-\mu \right)
,n->\infty .$Indeed, since $R\left( t\right) =o\left( t-\mu \right) ,\forall
\epsilon >0,\exists \delta >0,s.t.\forall t\in \left( \mu -\delta ,\mu
+\delta \right) /\left\{ \mu \right\} ,\left\vert \frac{R\left( t\right) }{%
t-\mu }\right\vert <\epsilon .$

Since $Y_{n}\overset{P}{\rightarrow }\mu ,$ $\forall \epsilon _{1}\exists
N,s.t.\forall n>N,P\left( \left\vert Y_{n}-\mu \right\vert \geq \delta
\right) <\epsilon _{1}.$

$P\left( \left\vert \frac{R\left( Y_{n}\right) }{Y_{n}-\mu }\right\vert
>\epsilon \right) =P\left( \left\vert \frac{R\left( Y_{n}\right) }{Y_{n}-\mu 
}\right\vert >\epsilon ,\left\vert Y_{n}-\mu \right\vert \geq \delta \right)
+P\left( \left\vert \frac{R\left( Y_{n}\right) }{Y_{n}-\mu }\right\vert
>\epsilon ,\left\vert Y_{n}-\mu \right\vert <\delta \right) \leq $

$P\left( \left\vert Y_{n}-\mu \right\vert \geq \delta \right) +P\left(
\left\vert \frac{R\left( Y_{n}\right) }{Y_{n}-\mu }\right\vert >\epsilon
,\left\vert Y_{n}-\mu \right\vert <\delta ,\left\vert \frac{R\left(
Y_{n}\right) }{Y_{n}-\mu }\right\vert <\epsilon \right) =P\left( \left\vert
Y_{n}-\mu \right\vert \geq \delta \right) <\epsilon _{1}$

$\implies \underset{n->\infty }{\lim }P\left( \left\vert \frac{R\left(
Y_{n}\right) }{Y_{n}-\mu }\right\vert >\epsilon \right) =0\implies $ $%
R\left( Y_{n}\right) =o_{p}\left( Y_{n}-\mu \right) ,n->\infty .$

And it is easy to verify that $o_{p}\left( Y_{n}-\mu \right) =\left(
Y_{n}-\mu \right) o_{p}\left( 1\right) .$

$\sqrt{n}R\left( Y_{n}\right) =\sqrt{n}o_{p}\left( Y_{n}-\mu \right) =\sqrt{n%
}\left( Y_{n}-\mu \right) o_{p}\left( 1\right) ,$since $\sqrt{n}\left(
Y_{n}-\mu \right) $ converges in distribution, by Slutsky's Thm, $\implies 
\sqrt{n}R\left( Y_{n}\right) \overset{d}{\rightarrow }0,$and by Problem 5.42 
$\implies \sqrt{n}R\left( Y_{n}\right) \overset{p}{\rightarrow }0.$

Then in the proof of Delta Method,$\sqrt{n}\left( g\left( Y_{n}\right)
-g\left( \theta \right) \right) =\left[ \sqrt{n}\left( Y_{n}-\theta \right) %
\right] g^{\prime }\left( \theta \right) +\sqrt{n}R\left( Y_{n}\right) .$

Also by Slutsky's Thm $\sqrt{n}\left( g\left( Y_{n}\right) -g\left( \theta
\right) \right) \overset{d}{\rightarrow }N\left( 0,g^{\prime }\left( \theta
\right) ^{2}\sigma ^{2}\right) .$

Problem 5.44

$\left( a\right) $ By CLT, $\frac{Y_{n}-E\left( Y_{n}\right) }{\sqrt{%
Var\left( Y_{n}\right) }}->N\left( 0,1\right) $ in distribution. Since $%
E\left( Y_{n}\right) =E\left( X_{1}\right) =p,Var\left( Y_{n}\right) =\frac{%
E\left( X_{1}\right) }{n}=\frac{p\left( 1-p\right) }{n}$

$\implies \frac{\sqrt{n}\left( Y_{n}-p\right) }{\sqrt{p\left( 1-p\right) }}%
\overset{d}{\rightarrow }N\left( 0,1\right) \implies \sqrt{n}\left(
Y_{n}-p\right) \overset{d}{\rightarrow }N\left( 0,p\left( 1-p\right) \right)
.$

$\left( b\right) $ Let $g\left( \theta \right) =\theta \left( 1-\theta
\right) ,$which has non-zero derivative 1-2$\theta $ at $\theta =p$, since p$%
\neq \frac{1}{2}.$

Then by Delta Method, $\sqrt{n}\left( g\left( Y_{n}\right) -g\left( p\right)
\right) \overset{d}{\rightarrow }N\left( 0,g^{\prime 2}\left( p\right)
p\left( 1-p\right) \right) $

That is $\sqrt{n}\left( Y_{n}\left( 1-Y_{n}\right) -p\left( 1-p\right)
\right) \overset{d}{\rightarrow }N\left( 0,\left( 1-2p\right) ^{2}p\left(
1-p\right) \right) $

$\left( c\right) $ For $p=\frac{1}{2},$Second Order Delta Method can be used
to derive 

$n\left( g\left( Y_{n}\right) -g\left( p\right) \right) \overset{d}{%
\rightarrow }\frac{1}{2}g^{\prime \prime }\left( p\right) ^{2}p\left(
1-p\right) \chi _{1}^{2},$that is $n\left( Y_{n}\left( 1-Y_{n}\right) -\frac{%
1}{4}\right) \overset{d}{\rightarrow }-\frac{1}{4}\chi _{1}^{2}.$

4. Let Xn\symbol{126}Beta$\left( \frac{1}{n},\frac{1}{n}\right) $. Show that
there exists r.v. X s.t. Xn--\TEXTsymbol{>}X in distribution,

and give the distribution of X explicitly.

Firstly we calculate the characteristic function of Beta distribution $%
B\left( p,q\right) $:

\FRAME{dtbpF}{4.4996in}{1.4762in}{0pt}{}{}{Figure}{\special{language
"Scientific Word";type "GRAPHIC";display "USEDEF";valid_file "T";width
4.4996in;height 1.4762in;depth 0pt;original-width 9.0312in;original-height
3.5518in;cropleft "0";croptop "1";cropright "1";cropbottom "0";tempfilename
'NWBCB40S.wmf';tempfile-properties "XPR";}}

For $p=q=\frac{1}{n},$the coefficient of power series reduces to $\frac{%
\Gamma \left( \frac{2}{n}\right) \Gamma \left( \frac{1}{n}+k\right) }{\Gamma
\left( \frac{1}{n}\right) \Gamma \left( \frac{2}{n}+k\right) }=\frac{\overset%
{k-1}{\underset{i=0}{\Pi }}\left( \frac{1}{n}+i\right) }{\overset{k-1}{%
\underset{i=0}{\Pi }}\left( \frac{2}{n}+i\right) }->1,$as $n->\infty .$ If
the limit process n-\TEXTsymbol{>}$\infty $ is changable with $\sum \qquad ($%
guaranteed by the uniform convergence of dominating series$\underset{k=0}{%
\overset{\infty }{\sum }}\frac{1}{k!})$ , then $\underset{n->\infty }{\lim }%
f_{X_{n}}\left( t\right) =\underset{k=0}{\overset{\infty }{\sum }}\frac{%
\left( it\right) ^{k}}{k!}=e^{it},$which is 

the characteristic function of a constant as r.v. $1.$

By the Continuous Thm, $X_{n}\overset{d}{\rightarrow }1.$

\end{document}
