
\documentclass{article}
%%%%%%%%%%%%%%%%%%%%%%%%%%%%%%%%%%%%%%%%%%%%%%%%%%%%%%%%%%%%%%%%%%%%%%%%%%%%%%%%%%%%%%%%%%%%%%%%%%%%%%%%%%%%%%%%%%%%%%%%%%%%%%%%%%%%%%%%%%%%%%%%%%%%%%%%%%%%%%%%%%%%%%%%%%%%%%%%%%%%%%%%%%%%%%%%%%%%%%%%%%%%%%%%%%%%%%%%%%%%%%%%%%%%%%%%%%%%%%%%%%%%%%%%%%%%
\usepackage{amsmath}

\setcounter{MaxMatrixCols}{10}
%TCIDATA{OutputFilter=LATEX.DLL}
%TCIDATA{Version=5.00.0.2552}
%TCIDATA{<META NAME="SaveForMode" CONTENT="1">}
%TCIDATA{Created=Saturday, December 26, 2015 20:52:30}
%TCIDATA{LastRevised=Sunday, December 27, 2015 00:00:29}
%TCIDATA{<META NAME="GraphicsSave" CONTENT="32">}
%TCIDATA{<META NAME="DocumentShell" CONTENT="Standard LaTeX\Blank - Standard LaTeX Article">}
%TCIDATA{CSTFile=40 LaTeX article.cst}

\newtheorem{theorem}{Theorem}
\newtheorem{acknowledgement}[theorem]{Acknowledgement}
\newtheorem{algorithm}[theorem]{Algorithm}
\newtheorem{axiom}[theorem]{Axiom}
\newtheorem{case}[theorem]{Case}
\newtheorem{claim}[theorem]{Claim}
\newtheorem{conclusion}[theorem]{Conclusion}
\newtheorem{condition}[theorem]{Condition}
\newtheorem{conjecture}[theorem]{Conjecture}
\newtheorem{corollary}[theorem]{Corollary}
\newtheorem{criterion}[theorem]{Criterion}
\newtheorem{definition}[theorem]{Definition}
\newtheorem{example}[theorem]{Example}
\newtheorem{exercise}[theorem]{Exercise}
\newtheorem{lemma}[theorem]{Lemma}
\newtheorem{notation}[theorem]{Notation}
\newtheorem{problem}[theorem]{Problem}
\newtheorem{proposition}[theorem]{Proposition}
\newtheorem{remark}[theorem]{Remark}
\newtheorem{solution}[theorem]{Solution}
\newtheorem{summary}[theorem]{Summary}
\newenvironment{proof}[1][Proof]{\noindent\textbf{#1.} }{\ \rule{0.5em}{0.5em}}


\begin{document}


\bigskip 10.5$\left( 1\right) $ Let $\sigma ^{\prime }=\sigma /\sqrt{n}$

$Var\frac{\sqrt{n}}{\overline{X_{n}}}=nVar\frac{1}{\overline{X_{n}}}=n\left(
\int_{\left\vert x\right\vert >\epsilon }\frac{1}{\sqrt{2\pi }\sigma
^{\prime }}\frac{1}{x^{2}}e^{-\frac{\left( x-\mu \right) ^{2}}{2\sigma
^{\prime 2}}}dx-\left( \int_{\left\vert x\right\vert >\epsilon }\frac{1}{%
\sqrt{2\pi }\sigma ^{\prime }}\frac{1}{x}e^{-\frac{\left( x-\mu \right) ^{2}%
}{2\sigma ^{\prime 2}}}dx\right) ^{2}\right) $

To show under the limit $\epsilon \rightarrow \infty ,Var\frac{1}{\overline{%
X_{n}}}\rightarrow \infty .$

Notice that $\int_{\left\vert x\right\vert >\epsilon }\frac{1}{\sqrt{2\pi }%
\sigma ^{\prime }}\frac{1}{x^{2}}e^{-\frac{\left( x-\mu \right) ^{2}}{%
2\sigma ^{\prime 2}}}dx=o\left( \frac{1}{\epsilon }\right) $

$\left( \int_{\left\vert x\right\vert >\epsilon }\frac{1}{\sqrt{2\pi }\sigma
^{\prime }}\frac{1}{x}e^{-\frac{\left( x-\mu \right) ^{2}}{2\sigma ^{\prime
2}}}dx\right) ^{2}=o\left( \log \epsilon \right) .$

$\frac{1}{\epsilon }-\log \epsilon =\frac{1-\epsilon \log \epsilon }{%
\epsilon },$By L'Hostipal Rule, $\frac{1-\epsilon \log \epsilon }{\epsilon }%
\rightarrow \infty .$

$\left( 2\right) $ from $\left( 1\right) $ we know that $Var\left(
T_{n}\right) <\infty $ if we delete the interval $\left( -\delta ,\delta
\right) $ 

\bigskip from the sample space. 

$\left( 3\right) $ If $\mu \neq 0,$ $P\left( \left\vert T_{n}\right\vert
<\delta \right) =P\left( \left\vert \overline{X_{n}}\right\vert >\frac{\sqrt{%
n}}{\delta }\right) =P\left( \left\vert \frac{\sigma Z}{\sqrt{n}}+\mu
\right\vert >\frac{\sqrt{n}}{\delta }\right) $

$=P\left( \left\vert \sigma Z+\sqrt{n}\mu \right\vert >\frac{n}{\delta }%
\right) <P\left( \left\vert \sigma Z\right\vert >\frac{n}{\delta }-\sqrt{n}%
\mu \right) \rightarrow 0.$

10.6 $\left( 1\right) $Since we observe $Y_{n}\sim N\left( 0,1\right) 
$ with probability $p_{n},Y_{n}\sim N\left( 0,\sigma _{n}^{2}\right) $
with probability $1-p_{n}.$

$E\left( Y_{n}\right) =p_{n}E\left( Z\right) +\left( 1-p_{n}\right) E\left(
\sigma _{n}Z\right) ,$where $Z\sim N\left( 0,1\right) \implies
E\left( Y_{n}\right) =0$

$Var\left( Y_{n}\right) =E\left( Y_{n}^{2}\right) =p_{n}E\left( Z^{2}\right)
+\left( 1-p_{n}\right) E\left( \sigma _{n}Z\right) ^{2}=p_{n}+\left(
1-p_{n}\right) \sigma _{n}^{2}$

$\left( 2\right) P\left( Y_{n}<a\right) =p_{n}P\left( Z<a\right) +\left(
1-p_{n}\right) P\left( \sigma _{n}Z<a\right) ,$as $p_{n}\rightarrow 1,$

$\sigma _{n}\rightarrow \infty ,P\left( \sigma _{n}Z<a\right) =P\left( Z<%
\frac{a}{\sigma _{n}}\right) \rightarrow \frac{1}{2}\implies $

$P\left( Y_{n}<a\right) =P\left( Z<a\right) .Hence,$ $Y_{n}\rightarrow
N\left( 0,1\right) .$

10.7 $\sqrt{n}\left( \hat{\theta}-\theta \right) \overset{d}{\rightarrow }%
N\left( 0,\frac{1}{I\left( \theta \right) }\right) $

By Delta's Method, $\sqrt{n}\left( \tau \left( \hat{\theta}\right) -\tau
\left( \theta \right) \right) \overset{d}{\rightarrow }N\left( 0,\frac{\tau
^{\prime 2}\left( \theta \right) }{I\left( \theta \right) }\right) .$

$\tau \left( \hat{\theta}\right) -\tau \left( \theta \right) =\frac{\tau
\left( \hat{\theta}\right) -\tau \left( \theta \right) }{\frac{\tau ^{\prime
}\left( \theta \right) }{\sqrt{I\left( \theta \right) }}/\sqrt{n}}\frac{\tau
^{\prime }\left( \theta \right) }{\sqrt{I\left( \theta \right) n}},\frac{%
\tau ^{\prime }\left( \theta \right) }{\sqrt{I\left( \theta \right) n}}%
\overset{P}{\rightarrow }0$ by definition. 

By Slusky's Thm, $\tau \left( \hat{\theta}\right) \overset{P}{\rightarrow }%
\tau \left( \theta \right) $

\end{document}
