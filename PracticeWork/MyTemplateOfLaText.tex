\documentclass[10pt]{article}
\newcounter{myCounter}
\newcommand{\head}[1]{\textbf{#1}}
\newcommand{\keyword}[2][\bfseries]{{#1#2}}
\usepackage{CJK}%preamble part
\usepackage{graphicx}
\usepackage{sidecap}
\usepackage[a4paper, inner=1.5cm, outer=3cm, top=2cm, bottom=3cm, bindingoffset=1cm]{geometry}
\usepackage{epstopdf}
\usepackage{array}
\usepackage{amsmath}
\usepackage{bm}
\usepackage{booktabs}
\setlength{\extrarowheight}{4pt}
\begin{document}


\begin{CJK}{GBK}{song}
\title{热传导方程的数值求解练习} % This is the title
\author{数33 赵丰 2013012178}
\date{Auguest 26,2016}
\maketitle
%\section{Abstract}
%\section{introduction}
\section{presentation}
\setlength{\parindent}{2em}
考虑一维热传导方程:
\begin{gather*}
  u_t=u_xx+f(x,t),\quad 0<x<1,t>0 \quad f(x,t)=(\pi^2-0.1)e^{-0.1t}\sin(\pi x) \\
  u(0,t)=0,u(1,t)=0,  \\
  u(x,0)=\sin(\pi x)
\end{gather*}
其解析解为:
$$ u(x,t)=e^{-0.1t}\sin(\pi x) $$
关于绝对稳定性的讨论可以推广到高阶常微分方程组,其结论是对于试验方程
$$ y'=Ay,A\in \bm{R}^{m\times n}$$
求解方程组的方法绝对稳定的必要条件是对A的所有特征值$\lambda,h\lambda$都要落在该方法的绝对稳定区间内。
将热传导方程在空间上作离散,可得一高维的常微分方程组
$$ u_t=-\frac{1}{\Delta x^2}T_N u +f $$
由于空间步长 $\Delta x$很小时,系数矩阵的特征值会是一个很负的数,对欧拉法而言,其绝对稳定区域为以
-1为圆心,1为半径的圆,因此步长h必须取得很小才能保证绝对稳定性。但对于隐式欧拉法和梯形法,由于整个负半平面
都是它的绝对稳定区域,因此步长的选取不受绝对稳定性的限制。
对欧拉法,绝对稳定性要求
$$ h<\frac{\Delta x^2}{2} $$
数值实验表明,在$\Delta x=\frac{1}{32}$的情况下,分别取$h=4\times 10^{-4}$ 和
$h=5\times 10^{-4}$,后者在t=1时波形图仍有较高的精度,但对前者则相差甚远。
这与 阀值 $ 4.88 \times 10^{-4} $ 稳合。
下表给出了不同空间步长用梯形法,即\textbf{Crank-Nicolson}格式求解的部分数值结果,
其中误差是取该时刻空间上与真实解的最大距离。
\begin{table}[!ht]
\centering
\caption{\textbf{Crank-Nicolson}格式求解热导方程}
\begin{tabular}{lllll}
\toprule[1.5pt]
  & & h=0.1,\textbf{Error} & & \\
\cmidrule(r){1-1} \cmidrule(l){2-5}
t & N=4&N=8&N=16&N=32\\
0.3 & 0.0299 & 0.0095 & 0.0027 & 7.10E-4\\
0.6 & 0.0303 & 0.0096 & 0.0027 & 7.17E-4\\
0.9 & 0.0295 & 0.0093 & 0.0026 & 6.97E-4\\
\bottomrule[1.5pt]
\end{tabular}
\end{table}
由上表可以看到,当空间步长取得过大(N=4)与时间步长不匹配时,会造成较大的求解误差。
下表给出了隐式欧拉法与梯形法求解结果的比较。
% Table generated by Excel2LaTeX from sheet 'Sheet3'
\begin{table}[htbp]
  \centering
  \caption{h=0.1,N=32隐式欧拉法与梯形法求解结果比较}
    \begin{tabular}{rrr}
    \hline
  \multicolumn{1}{l}{t} &  \multicolumn{1}{l}{隐式欧拉法} & \multicolumn{1}{l}{梯形法} \\
  \hline
 0.2 &  0.00059143 & 0.000659 \\
 0.4&   0.00072964 & 0.000722 \\
 0.6&   0.00075319 & 0.000717 \\
 0.8&   0.00074791 & 0.000704 \\
 1.0&   0.00073554 & 0.00069 \\
 \hline
    \end{tabular}%
\end{table}%
  由上表可以看出,两种方法差别不大。
\end{CJK}
\end{document}